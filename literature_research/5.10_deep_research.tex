Polyreactivity and Polyspecificity Signatures: CHO-Lysate Determinants and Entropy-Linked Risks
Definition of Polyreactivity

Polyreactivity refers to an antibody’s tendency to bind multiple unrelated antigens with low affinity (sometimes called “promiscuous” or non-specific binding)
pubmed.ncbi.nlm.nih.gov
. This is distinct from polyspecificity, which involves off-target binding to a limited set of discrete non-cognate proteins (often with somewhat higher affinity)
pubmed.ncbi.nlm.nih.gov
. In practical terms, polyreactive antibodies exhibit “stickiness” – they can weakly interact with a broad spectrum of molecules (DNA, proteins, lipids, etc.) that are unrelated to their intended target. Assays for polyreactivity typically measure binding to complex mixtures of molecules. For example, binding to cell lysates (e.g. Chinese hamster ovary CHO cell membrane lysate) or to surrogate reagents like polyanionic heparin, nucleic acids, or unrelated proteins (such as insulin or ovalbumin) is used as a readout for non-specific interactions
pubmed.ncbi.nlm.nih.gov
pubmed.ncbi.nlm.nih.gov
. High signals in these assays indicate a polyspecificity liability, meaning the antibody may have developability issues due to non-specific binding.

Mechanistic Model of Polyreactivity

Multiple molecular features drive antibody polyreactivity:

Electrostatic Attraction: An excess of positive charge on an antibody’s surface (especially in the complementarity-determining regions, CDRs) predisposes it to bind negatively charged species non-specifically
pubmed.ncbi.nlm.nih.gov
pubmed.ncbi.nlm.nih.gov
. Many cell-derived molecules (membrane phospholipids, glycosaminoglycans, nucleic acids, acidic glycoproteins) carry anionic charges. Antibodies with high CDR pI or clustered lysine/arginine residues can experience ionic interactions with such molecules, leading to low-affinity binding
patents.google.com
. Indeed, an imbalance toward positive charge in the variable domains is a well-known cause of “sticky” behavior
pubmed.ncbi.nlm.nih.gov
. These electrostatic interactions are nonspecific and can occur with many negatively charged partners in cell lysates or extracellular matrix
pubmed.ncbi.nlm.nih.gov
.

Hydrophobic Patches: Exposed hydrophobic residues (e.g. tryptophan, phenylalanine, tyrosine, isoleucine) on the antibody surface can mediate non-specific binding via the hydrophobic effect
pubmed.ncbi.nlm.nih.gov
. Regions in the paratope that are solvent-exposed and rich in aromatic or aliphatic residues tend to stick to other hydrophobic regions of proteins or lipid membranes
pubmed.ncbi.nlm.nih.gov
. In complex mixtures (like cell lysate), partially folded proteins or membrane fragments present hydrophobic patches that can capture antibodies with complementary hydrophobic surfaces
patents.google.com
. Thus, antibodies with large aromatic CDR content or high hydrophobic surface area are prone to polyreactivity. Notably, these “greasy” patches often act together with charge effects – e.g. an antibody with both high positive charge and hydrophobic CDR regions has an elevated risk of polyspecific binding
pubmed.ncbi.nlm.nih.gov
.

Conformational Flexibility (Entropy): Antibodies that have flexible or disordered CDR loops can sample many conformations (“microstates”), which may enable them to accommodate diverse unrelated ligands. High sequence entropy in certain CDR positions (meaning a wide variety of amino acids observed at that position among polyreactive clones) can correlate with this flexibility. In other words, polyreactive antibodies may come from sequence families that tolerate more variation (less constrained), suggesting a dynamic binding site
pubmed.ncbi.nlm.nih.gov
pmc.ncbi.nlm.nih.gov
. This flexibility can broaden the range of low-affinity interactions (“conformational promiscuity”). Moreover, certain external stressors can induce flexibility: for example, exposing an antibody to acidic pH or introducing arginine mutations in the CDRs has been shown to increase conformational lability and polyreactivity
pubmed.ncbi.nlm.nih.gov
. In fact, acidic pH treatments (such as the low pH used in Protein A chromatography elution or viral inactivation steps) can transiently unfold or alter antibody domains, leading to induced polyreactivity upon return to neutral pH
pubmed.ncbi.nlm.nih.gov
. Similarly, Arg substitutions in the antigen-binding region – sometimes introduced in engineering for stability – can paradoxically create new nonspecific binding contacts
pubmed.ncbi.nlm.nih.gov
. These observations underscore that an antibody’s conformational ensemble and entropy contribute to polyreactive behavior.

In summary, polyreactivity arises primarily from surface property imbalances (excess positive charge, excess hydrophobic area) on an antibody – predominantly in the heavy chain CDRs – combined with structural flexibility that allows the antibody to adapt to various nonspecific ligands
pubmed.ncbi.nlm.nih.gov
. These mechanisms are largely additive: e.g., an antibody with a highly positively charged and hydrophobic heavy chain HCDR3 that is also very flexible is extremely likely to be polyspecific, whereas a well-balanced, rigid antibody is usually specific.

CHO Lysate Binding Determinants (PSR/PSP Context)

One common method to evaluate polyreactivity is to test antibody binding to complex cell lysates. CHO cells (Chinese Hamster Ovary cells) are frequently used to prepare such lysate-based reagents, as they provide a rich mixture of proteins, nucleic acids, lipids, and glycans that can reveal nonspecific binding tendencies
patents.google.com
patents.google.com
. Key aspects include:

Reagent Composition: In a typical polyspecificity assay (often termed PSR for “Polyspecificity Reagent”), soluble membrane proteins (SMP) are extracted from cell membranes (e.g. CHO cell membranes) using detergents
patents.google.com
. This yields a heterogeneous mixture of biomolecules: membrane proteins (many heavily glycosylated and negatively charged), lipids/lipoproteins, nucleic acids, and other cytosolic proteins
patents.google.com
patents.google.com
. For example, one described PSR preparation includes biotinylated SMP fractions from CHO cells or from insect cells (Sf9), and sometimes a soluble cytosolic protein fraction (SCP) from Sf9, each containing hundreds of different molecules
patents.google.com
. These mixtures present a broad array of potential binding partners. In practice, the PSR mixture is immobilized (e.g. on beads or via biotin-streptavidin) and the test antibody’s binding is detected (by flow cytometry or ELISA readout of bound antibody)
researchgate.net
. A high binding signal to CHO lysate indicates the antibody is polyspecific, engaging many components non-specifically.

Antibody Features Driving Lysate Binding: Analyses of numerous antibodies have shown that certain sequence/structure features correlate strongly with high lysate binding. The primary drivers are:

Net Cationic Charge in CDRs – Antibodies with more positively charged residues (Lys, Arg) than negatives in their CDRs tend to give high PSR binding
pubmed.ncbi.nlm.nih.gov
. Localized positive charge patches (e.g. a cluster of basic residues on one face of the CDR loop) are especially predictive of sticky behavior
patents.google.com
.

CDR Hydrophobic Content – A high number of aromatic residues (Trp, Tyr, Phe) or other hydrophobics in the CDRs is another risk factor. These residues can mediate nonspecific hydrophobic interactions with the diverse proteins and lipids in the lysate
pubmed.ncbi.nlm.nih.gov
. Antibodies with large exposed hydrophobic patches (often quantified by hydrophobic interaction chromatography or other methods) generally show higher polyreactivity.

Heavy Chain Dominance (HCDR3 length/composition) – The heavy chain, and in particular the third heavy chain loop (HCDR3), often contributes most of the problematic charge or hydrophobic features
pubmed.ncbi.nlm.nih.gov
. A study of >300,000 antibody sequences found that polyreactivity is governed primarily by the heavy chain CDRs, which frequently contain the high positive charge and/or hydrophobic motifs responsible
pubmed.ncbi.nlm.nih.gov
. Long HCDR3 loops are often culprits, as they can harbor more residues and unusual sequences (including potential polyreactive motifs like Arg-Tyr rich segments
pmc.ncbi.nlm.nih.gov
). In practice, many known polyreactive mAbs have an unusually long or Arg-rich HCDR3.

Matrix (Lysate) Factors: The composition of the CHO lysate itself biases the interactions toward certain modes. CHO cell membranes, for instance, contain abundant negatively charged glycans (sialic-acid-rich) and sulfated glycosaminoglycans on proteoglycans – these readily attract any antibody with a cationic patch
pubmed.ncbi.nlm.nih.gov
. The lysate also contains nucleic acids (DNA/RNA are polyanionic) which can bind positively charged antibodies. Meanwhile, detergent-solubilized membrane proteins and lipids in the mix present hydrophobic surfaces, which will trap antibodies exposing hydrophobic CDR regions
patents.google.com
. Importantly, these interactions in the PSR assay are not specific lock-and-key bindings to a unique epitope, but rather generic stickiness. The patent describing PSR reagents emphasizes that the antibody-lysate interactions “are believed to [be] driven by non-specific ionic, electrostatic, hydrophobic or other such interactions” rather than any true antigen-antibody recognition
patents.google.com
. In short, the complexity of the lysate (many potential binding partners) and the biochemical makeup of its components (overall negative charge, presence of hydrophobes) combine to make CHO lysate an effective “sticky-surface” probe for antibody polyreactivity.

Interpreting CHO Lysate Binding: Antibodies that bind strongly to CHO SMP lysate in a PSR assay almost always also bind other unrelated substrates (DNA, heparin, etc.), indicating a promiscuous binding tendency. Conversely, antibodies engineered to have balanced charge and minimal hydrophobic patches typically show low CHO lysate binding and are more specific
pubmed.ncbi.nlm.nih.gov
. Thus, CHO lysate binding is used in many labs as an early screen: antibodies above a certain median fluorescence intensity (MFI) or percentile in this assay may be flagged as polyspecific and risky for development.

PSR vs. PSP Assays and Signatures

There are multiple related assay formats to measure an antibody’s polyspecificity. Two notable ones are the Polyspecificity Reagent (PSR) assay and the more recent PolySpecificity Particle (PSP) assay:

Traditional PSR Assay: The PSR assay was popularized by Adimab and others
patents.google.com
patents.google.com
. In its original implementation, antibodies (as Fab or full IgG) can be displayed on yeast surface or tested in solution, and incubated with a PSR mixture (like biotinylated CHO membrane proteins). Binding is detected via flow cytometry or ELISA by measuring how much of the PSR material attaches to the antibody
patents.google.com
patents.google.com
. This yields a quantitative readout, e.g. median fluorescence. High PSR binding correlates with other indicators of nonspecificity and often with in vivo clearance issues
dspace.mit.edu
. In fact, one study showed a clear relationship between an antibody’s PSR binding signal and its clearance rate in mice – high PSR “sticky” antibodies cleared faster
dspace.mit.edu
. The PSR assay typically requires preparing the complex lysate reagent and (in some implementations) displaying antibodies on yeast or beads, which can be proprietary or labor-intensive.

PSP Assay (PolySpecificity Particle): The PSP assay is a newer, high-sensitivity method introduced to overcome some PSR limitations
researchgate.net
. In the PSP assay, the test antibodies are first immobilized on standard magnetic beads (coated with Protein A to capture IgGs)
researchgate.net
. These antibody-coated beads are then incubated with various polyspecificity reagents – for example, soluble CHO membrane extract, or other complex mixtures
researchgate.net
. Nonspecific binding is detected via flow cytometry by fluorescently labeling what binds to the antibody-coated beads
researchgate.net
. The advantages of PSP are that it uses off-the-shelf reagents (magnetic beads and purified antibodies) rather than requiring specialized yeast display, and it works for any soluble antibody format including bispecifics
researchgate.net
researchgate.net
. According to Makowski et al. (2021) who developed it, the PSP assay can discriminate different polyspecificity levels with high sensitivity
pmc.ncbi.nlm.nih.gov
researchgate.net
. Essentially, it’s a bead-based analog of the PSR binding measurement, amenable to high-throughput screening.

Assay Performance and Correlation: Studies have found that PSR and PSP assays strongly agree. For example, one high-throughput study tested ~200 antibodies with both CHO-SMP (as in PSR) and a simpler surrogate (ovalbumin, OVA) in a bead-based format, and found a good correlation between the two reagents
researchgate.net
researchgate.net
. Most antibodies that were “sticky” to CHO lysate were also sticky to OVA, and vice versa, allowing OVA to serve as a convenient initial proxy for polyreactivity
researchgate.net
. The PSP assay, using flow cytometry readouts, is very sensitive even when little material is available, and it does not require large quantities of cell lysate. As a result, many groups now employ a panel of polyspecificity assays: for instance, a low-volume PSP screen against a simple reagent (like OVA or DNA) to triage many candidates, followed by confirmatory testing on the more complex CHO lysate PSR assay for finalists
researchgate.net
researchgate.net
. A strong response in any of these assays (PSR or PSP) typically flags the antibody for further engineering or deselection.

Clustering with Other Developability Assays: Polyspecificity assays (whether PSR or PSP) tend to cluster with other charge-related interaction assays in terms of what they detect
pmc.ncbi.nlm.nih.gov
. For instance, studies performing principal component or clustering analysis on developability data have found that PSR, cross-interaction chromatography (CIC), affinity-capture self-interaction (AC-SINS/CSI), and viral particle binding (BVP) all group together – they all primarily sense non-specific, charge-driven interactions
pmc.ncbi.nlm.nih.gov
pmc.ncbi.nlm.nih.gov
. In contrast, assays like hydrophobic interaction chromatography (HIC) or salt-gradient self-association (SGAC) cluster separately, as they measure hydrophobicity-dominated behavior
pmc.ncbi.nlm.nih.gov
. For example, one analysis identified a “polyreactivity/charge” cluster containing PSR, CIC, CSI, etc., and a distinct “hydrophobicity” cluster containing HIC and a monolayer adsorption assay (SMAC)
pmc.ncbi.nlm.nih.gov
. This means an antibody that fails PSR is also likely to show high retention in CIC or high self-interaction, whereas an antibody that fails HIC might be a different subset (hydrophobic but not highly charged). However, extreme cases can blur these lines: an antibody with both high charge and hydrophobe content might read out in all assays. Notably, at very high salt concentrations (as used in HIC), even charge-mediated effects can emerge (salt can shield charges, but certain strong ionic interactions or structural changes might still influence HIC behavior). In general, though, PSR/PSP results are most comparable to other “stickiness” assays like CIC and BVP, reflecting primarily electrostatic stickiness, whereas HIC reflects hydrophobic stickiness
pmc.ncbi.nlm.nih.gov
. This orthogonality is why developability screening often includes at least one assay of each type (charge-based and hydrophobic-based) to capture all liabilities.

Entropy-Linked Polyspecificity Risks

Beyond obvious sequence features like charge and hydrophobicity, more subtle properties like sequence entropy and loop dynamics can influence polyreactivity:

Sequence Entropy Signatures: Using large databases of polyreactive vs non-polyreactive antibody sequences, researchers have looked for patterns in sequence variability. One study (Boughter et al., 2020) applied information theory metrics and found that polyreactive antibodies exhibit differences in CDR sequence diversity and inter-loop coupling
pubmed.ncbi.nlm.nih.gov
pubmed.ncbi.nlm.nih.gov
. Interestingly, the overall Shannon entropy (a measure of diversity at each sequence position) was not drastically different between polyreactive and non-polyreactive sets when averaged globally
pubmed.ncbi.nlm.nih.gov
, implying both sets can be diverse. However, local differences emerged: certain positions (particularly in HCDR loops) showed higher correlated variability (“crosstalk”) in polyreactive antibodies
pubmed.ncbi.nlm.nih.gov
. This suggests that polyreactive antibodies may employ a wider array of solutions (amino-acid choices) at key binding positions, potentially giving them flexibility to bind multiple targets. In practical terms, an antibody family with very high sequence entropy at a given CDR position means that position isn’t crucial for a single specific epitope (since many amino acids are tolerated), which could correlate with the antibody being able to accommodate different epitopes – a polyspecific tendency. Thus, a broad sequence repertoire and lack of stringent conservation at certain paratope positions can be a risk marker.

Induced Polyreactivity (External Factors): Antibody polyreactivity is not always fixed – it can be induced or exacerbated by external conditions or sequence modifications. Low pH exposure is a prime example: antibodies that undergo acid treatments can temporarily become more polyreactive
pubmed.ncbi.nlm.nih.gov
. Empirically, antibodies purified with harsh acid elution (pH ~3) have shown higher non-specific binding in subsequent assays compared to the same antibodies purified under gentler conditions
pubmed.ncbi.nlm.nih.gov
pubmed.ncbi.nlm.nih.gov
. This is thought to occur because acid pH causes partial unfolding or alternative conformations (e.g. a “molten globule” state) that do not fully refold to the exact original state, leaving some hydrophobic or charged patches more exposed
pubmed.ncbi.nlm.nih.gov
. Similarly, arginine mutations in CDRs – sometimes introduced during affinity maturation or humanization – have been linked to increased polyreactivity
pubmed.ncbi.nlm.nih.gov
. Arginine is a curious residue: while often used as an excipient to suppress aggregation, within an antibody’s sequence Arg is highly basic and can form multiple hydrogen bonds. Replacing a smaller residue with Arg in a CDR could introduce a new positive patch or alter loop conformation, leading to new unintended interactions. A recent review highlights that polyreactivity can emerge during somatic hypermutation in germinal centers, where B cells occasionally acquire Arg or other mutations that increase flexibility or charge
pubmed.ncbi.nlm.nih.gov
. Takeaway: Certain conditions (acidic pH, specific mutations like Arg insertion) enhance antibody flexibility or expose new surfaces, thereby increasing nonspecific binding. Developers must be cautious with these factors – for instance, minimizing low-pH hold times during manufacturing and scrutinizing any CDR changes that add Arg or increase disorder.

Conformational Flexibility and Dynamics: Polyspecific antibodies often have dynamic binding sites. Structural studies (e.g. NMR or molecular dynamics simulations) have shown that some polyreactive antibodies exist in multiple conformations, even in unbound form
pmc.ncbi.nlm.nih.gov
. This inherent flexibility means the antibody can adjust and bind to different shapes – a beneficial trait for cross-reactive immune defense, perhaps, but detrimental for a therapeutic seeking a single target
pubmed.ncbi.nlm.nih.gov
pubmed.ncbi.nlm.nih.gov
. Flexibility is hard to quantify directly in sequence, but proxies include longer loop length (especially HCDR3) and lower predicted stability. Indeed, unusually long HCDR3 loops (e.g. >20 amino acids) are often polyreactive unless stabilized by internal disulfides or motifs, because a long loop can flop into different orientations to bind various partners. Similarly, antibodies that are on the edge of stability (somewhat unstable) may sample partly unfolded states that stick to lots of things. Overall, entropy in the antibody’s structure (many accessible conformational states) correlates with polyspecific binding – this is an area of active research, connecting biophysical measurements with sequence-based features
pmc.ncbi.nlm.nih.gov
.

Quantitative Sequence-Level Predictors

A number of sequence-derived metrics have been proposed to flag antibodies with high polyreactivity risk. Many can be calculated directly from the antibody’s Fv sequence:

$\text{NetCDRCharge} \;=\; (\#\text{Lys} + \#\text{Arg}) \;-\; (\#\text{Asp} + \#\text{Glu})$

$\text{AbsCDRCharge} \;=\; (\#\text{Lys} + \#\text{Arg}) \;+\; (\#\text{Asp} + \#\text{Glu})$

$\text{AromaticCount}_{\mathrm{CDR}} \;=\; (\#\text{Phe} + \#\text{Trp} + \#\text{Tyr})$

$\text{CDR-pI} \;=\;$ Isoelectric point of the combined Fv (or just CDR regions)

$\text{MaxPositivePatch} \;=\;$ maximum localized positive surface potential (from structure or model)

$L_{\mathrm{HCDR3}} \;=\;$ length of the HCDR3 loop (amino acid count)

$\text{ArgFraction}_{\mathrm{CDR}} \;=\;$ fraction of CDR residues that are arginine

$H_{\mathrm{seq}}(i) \;=\; -\sum_{aa} p_{i}(aa)\log p_{i}(aa)$ (Shannon entropy at position $i$ across a sequence set)


These features address the factors discussed earlier. In practice, high NetCDRCharge (especially if strongly positive) is a red flag
pubmed.ncbi.nlm.nih.gov
. A large number of aromatics in CDRs or a very high hydrophobic patch score also correlates with polyreactivity
pubmed.ncbi.nlm.nih.gov
. Several studies have found Arginine content to be particularly important: polyreactive antibody sets had an increased prevalence of Arg (and also Tyr) compared to non-polyreactives
pmc.ncbi.nlm.nih.gov
. Arginine in CDRs contributes both positive charge and potential H-bonding stickiness, so counting Arg residues is useful. HCDR3 length is another indicator – while not all long HCDR3 antibodies are polyreactive, many problem antibodies have longer-than-average HCDR3 loops (which provide more opportunity for undesirable motifs or flexibility).

Another sophisticated approach is to use machine learning models trained on these features to predict polyreactivity. Recent work by Chen et al. (2024) developed a model using heavy-chain sequence features (charge, hydrophobicity, etc.) that successfully distinguished polyreactive antibodies
pubmed.ncbi.nlm.nih.gov
. This underscores that a combination of the above quantitative features can yield a predictive score. For example, an internal scoring model might weight Net Charge, Hydrophobic Patch and HCDR3 length to compute a “polyspecificity risk score.” Empirically, antibodies that exceed certain thresholds (e.g. Net CDR charge > +5, or >30% aromatic CDR residues, etc.) are much more likely to show high PSR/PSP readings
pubmed.ncbi.nlm.nih.gov
pmc.ncbi.nlm.nih.gov
. Such rules of thumb, however, have exceptions – thus the push toward ML models that can learn non-linear combinations of features.

Empirical Risk Signals: In summary, sequence features that often signal risk include: very high positive charge (and/or uneven charge distribution) in CDRs, a high count of aromatic/hydrophobic residues in the paratope, an imbalance in charge (lots of one charge without counter-balancing opposite charges), an unusually long or flexible HCDR3, an abundance of arginine in CDRs, and indications of high sequence entropy (positions that are not conserved at all among close variants). Antibodies hitting multiple of these criteria should be scrutinized closely, as they are prime candidates for polyspecificity liabilities.

Cross-Assay Expectations and Comparisons

Different developability assays highlight different liabilities. The expected behavior of a given antibody across a panel of assays can inform which biophysical factor is driving any issues:

Liability Feature	PSR / PSP (CHO lysate)	CIC / CSI / BVP (charge-driven assays)	HIC (hydrophobic interaction)
High CDR positive charge	High binding – strong signal due to attraction to negative lysate components
pubmed.ncbi.nlm.nih.gov
.	High retention / signal – tends to cause strong interaction in ion-exchange or cross-interaction chromatography
pubmed.ncbi.nlm.nih.gov
.	Low retention at low salt – highly charged antibodies may elute early in HIC (since they lack hydrophobicity), but unusually at very high salt they might show some interaction (salt can expose hydrophobic facets).
Exposed aromatic patch	High binding – hydrophobic patches contribute to nonspecific lysate binding
pubmed.ncbi.nlm.nih.gov
.	Moderate–High – context-dependent. Can increase self-interaction (especially in low salt or in BVP where hydrophobic contacts matter).	High retention – strong hydrophobic patches will greatly increase HIC retention (late elution)
sciencedirect.com
.
Balanced charge (no strong patch)	Low binding – well-balanced antibodies (e.g. equal pos/neg charges) give minimal PSR binding
pubmed.ncbi.nlm.nih.gov
.	Low – fewer nonspecific ionic interactions, lower CIC signals.	Mixed – charge balance itself doesn’t directly affect HIC, which cares more about hydrophobicity; however, highly polar antibodies (balanced and hydrophilic) might actually elute early in HIC (which is good).
High flexibility / entropy	High binding – flexible, partially disordered paratopes can accommodate many lysate ligands, increasing PSR/PSP signal
pubmed.ncbi.nlm.nih.gov
.	High – often correlates with self-interaction and aggregation propensity; such antibodies might show higher BVP or self-interaction (though not always).	Variable – if flexibility exposes hydrophobic regions, HIC will be high; if it primarily increases dynamics without hydrophobes, HIC might not be strongly affected.

Table: Qualitative expectations for how certain antibody features manifest in different developability assays. PSR/PSP and CIC/BVP tend to flag similar charge-mediated liabilities, whereas HIC flags hydrophobic liabilities
pmc.ncbi.nlm.nih.gov
.

In practice, one often observes that antibodies with, say, very high positive charge will fail both the CHO lysate binding test and the CIC (cross-interaction chromatography) test (both show strong interactions)
pubmed.ncbi.nlm.nih.gov
. Those with primarily a hydrophobic issue (but neutral charge) might pass CIC/PSR but fail HIC (high retention). Antibodies with multiple liabilities (e.g. high charge and high hydrophobicity) will fail in all assays – these are most problematic. By examining this assay pattern, developability scientists infer the dominant problem and can prioritize engineering strategies (e.g. if only HIC is failing, focus on removing hydrophobic patch; if PSR/CIC failing but HIC fine, focus on removing positive charge patch, etc.).

It’s worth noting that FcRn binding assays (not detailed in the table) also interact with these features: Excess positive charge can increase non-specific binding to the neonatal Fc receptor at neutral pH, which may show up as an abnormally strong FcRn affinity at pH7.4
pubmed.ncbi.nlm.nih.gov
. This can tie into faster clearance as well.

Pharmacokinetic (PK) and Clearance Linkage

Polyspecificity is not just an in vitro concern; it has direct implications for how antibodies behave in vivo. Antibodies with high polyreactivity often exhibit poor pharmacokinetics, mainly through faster clearance from circulation. The mechanisms are:

Endothelial/Membrane Binding and Uptake: Antibodies with exposed positive charges or hydrophobic patches tend to adhere nonspecifically to cell membranes in the body (for example, the endothelium of blood vessels or cellular surfaces in liver and kidney)
pubmed.ncbi.nlm.nih.gov
pubmed.ncbi.nlm.nih.gov
. This can trigger increased pinocytosis (non-specific cellular uptake). Normally, antibodies rely on the neonatal Fc receptor (FcRn) to recycle them and avoid degradation. However, if an antibody is stuck to cell surfaces or gets nonspecifically endocytosed at a high rate, the clearance pathways can be overwhelmed, leading to faster degradation and elimination
pubmed.ncbi.nlm.nih.gov
. Figure 2 of Cunningham et al. (2021) visually illustrates that “excess positive charge in antibody variable domains... [can] drive accelerated pinocytosis,” reducing PK half-life
pubmed.ncbi.nlm.nih.gov
.

FcRn Off-Target Interactions: Another factor is that a highly positively charged antibody may bind to FcRn even at neutral pH (where it normally should not)
pubmed.ncbi.nlm.nih.gov
. FcRn is meant to bind IgG in acidified endosomes (pH ~6) and release at pH 7.4. If an antibody partially sticks to FcRn at pH 7.4 (due to non-specific surface contacts, often charge-mediated), it can disrupt the normal recycling process, causing the antibody to be erroneously targeted for lysosomal degradation
pubmed.ncbi.nlm.nih.gov
. The cited review notes that charge imbalance can “increase the affinity of antibodies for FcRn at pH7.4, impairing recycling”
pubmed.ncbi.nlm.nih.gov
, thus shortening half-life.

Retention in Depots: Hydrophobic antibodies may also partition into hydrophobic environments (for instance, associating with albumin or lipoproteins, or even forming small aggregates). This can affect their volume of distribution and clearance. Hydrophobic or sticky antibodies might accumulate in clearance organs or be taken up by phagocytic cells more readily
pubmed.ncbi.nlm.nih.gov
.

The net effect is that polyreactive antibodies often have subpar PK profiles – higher clearance and sometimes nonlinear PK. For example, Avery et al. (2018) showed that antibodies with high nonspecific binding (via DNA/insulin binding assays and self-interaction assays) tended to have faster clearance in human FcRn-transgenic mice and in human trials
pubmed.ncbi.nlm.nih.gov
pubmed.ncbi.nlm.nih.gov
. Many companies have internal “rules” now to eliminate candidates that exceed certain polyspecificity or hydrophobicity thresholds precisely to avoid PK liabilities
pubmed.ncbi.nlm.nih.gov
pubmed.ncbi.nlm.nih.gov
. It’s not just about half-life: polyreactive antibodies can also have more off-target toxicity or cause false positives in safety assays by binding irrelevant targets. Therefore, controlling polyreactivity is crucial for a successful therapeutic antibody.

In summary, an antibody’s “developability” is tightly linked to its off-target binding profile: Charge imbalance and excess hydrophobicity, the hallmarks of polyreactivity, combine to reduce PK and bioavailability of therapeutic antibodies
pubmed.ncbi.nlm.nih.gov
. Ensuring a well-behaved PK often starts with ensuring a clean polyspecificity panel readout in vitro.

Engineering and Mitigation Strategies

When an antibody shows polyreactivity liabilities, several engineering interventions can be employed to improve its profile:

Neutralize Charged Patches: If an antibody has a cluster of basic residues in a CDR (especially HCDR3) that is contributing to a high positive charge patch, one strategy is to mutate some of those to more neutral or acidic amino acids. For example, Lys/Arg to Gln/Asn or Glu/Asp substitutions at solvent-exposed positions can significantly reduce nonspecific binding
pubmed.ncbi.nlm.nih.gov
. The goal is to bring the Net CDR Charge closer to zero and eliminate any highly localized charge hot-spots. It’s important to choose mutations that don’t disrupt antigen binding – often one can mutate peripheral CDR residues (not directly contacting the antigen) to tone down the charge. Balance is key: if an antibody has a very acidic patch, that can sometimes cause aggregation or other issues too, but typically the problematic case is a strongly basic patch, which is more common among human antibodies
pubmed.ncbi.nlm.nih.gov
. By judiciously introducing polar or acidic residues, one can retain specificity while reducing polyspecificity. In a published case, mutating two arginines to glutamines in an antibody’s CDR dramatically lowered its CHO cell binding without hurting affinity to its target (thus “despeciating” it).

Reduce Exposed Hydrophobic Residues: Identify CDR hydrophobic patches (e.g., a stretch of aromatic residues on a loop’s solvent face) and consider making them more polar. For instance, replace a Phe or Tyr with a serine or threonine if it is not critical for antigen binding, or mutate a Trp to a Tyr (tyrosine is slightly less hydrophobic and has a polar hydroxyl). In some cases, introducing an O-linked glycosylation sequon in a CDR (i.e., an NXS/T motif leading to glycosylation) has been used to mask a hydrophobic patch with a sugar chain – though this must be done carefully as it can affect binding. The principle is to retain only the hydrophobic/aromatic residues that are absolutely required for the paratope, and make the rest more hydrophilic. It was noted that retaining an aromatic core in the paratope is fine (for affinity), but exposed aromatics that aren’t buried in the antibody-antigen interface can usually be altered to reduce stickiness. By doing so, one can lower HIC retention and improve solubility, as well as reduce PSR binding
pubmed.ncbi.nlm.nih.gov
.

Lower Arginine Content: As mentioned, arginine in CDRs is a repeat offender for polyreactivity
pmc.ncbi.nlm.nih.gov
. If an antibody has multiple Arg residues, especially in HCDR3, one might swap some of them for lysine or for less basic residues. Lysine is also positively charged but lacks the guanidinium group’s multi-mode binding capability, and sometimes a Lys can suffice for antigen binding if an Arg was not strictly required (Lys is a bit less likely to cause nonspecific binding than Arg). Alternatively, if feasible, substitute Arg with polar glutamine or histidine (histidine is positively charged only at slightly acidic pH, so at physiological pH it’s less basic). Reducing Arg density has been empirically shown to reduce polyspecific binding
pubmed.ncbi.nlm.nih.gov
. (Of course, if Arg is part of a critical epitope contact, one must be cautious.) Often, Arg-rich motifs like ARGY or RRY in CDRs are a red flag; breaking them up can help.

Shorten or Stabilize HCDR3: A very long and flexible HCDR3 can be trimmed if the extra length is not needed for target binding. Sometimes antibodies from synthetic libraries have insertions or extensions that aren’t actually utilized for the antigen contact; removing these can reduce the floppy, sticky appendages on the antibody. In cases where HCDR3 must remain long (e.g., it forms the binding site), one can try to rigidify it. Introducing a disulfide bond within HCDR3 (if there’s a convenient pair of cysteine positions that can be added) has been used to lock a loop conformation and reduce flexibility. Another approach is incorporating Proline residues at certain positions to restrict the backbone flexibility (proline being a helix/loop breaker that freezes the phi angle). Conversely, adding a glycine can sometimes increase flexibility, which is usually not what we want – so generally avoid adding glycine unless trying to relieve a strain. By tuning the loop composition (e.g., Pro-Gly patterns or adding a small β-turn motif), one can sometimes reduce the entropic freedom of HCDR3, thereby reducing polyreactivity. The antibody’s structure may become more rigid and specific in its binding.

Process and Formulation Controls: Sometimes, no sequence change is needed – instead, how you handle the antibody can mitigate polyreactivity. For instance, as discussed, avoiding low pH exposure can prevent induced polyreactivity
pubmed.ncbi.nlm.nih.gov
. Many developers now use neutral pH elution buffers for Protein A chromatography (or use gentle acidic pH with arginine-containing buffers that prevent unfolding) to keep the antibody in its native state. Similarly, virus inactivation can be done by solvent/detergent methods or pasteurization instead of low pH, if polyspecificity is a big concern. If low pH can’t be avoided, rapidly neutralizing the solution and adding stabilizers (like arginine as an excipient, paradoxically, or glycine) might help the antibody refold correctly. On the formulation side, adding certain additives can reduce nonspecific self-interactions (e.g., polyols or amino acids in the formulation can shield sticky patches). While these don’t change the antibody’s sequence, they can reduce the manifestation of polyreactivity in vivo or in tests. Nonetheless, the preference is to engineer the antibody itself to be robust.

In applying these strategies, the priority is to eliminate the most egregious liabilities first. A common workflow is: if an antibody fails the PSR assay due to high positive charge, focus on charge mutations first (since charges are often easier to adjust without killing binding). If it fails due to hydrophobicity (high HIC retention), focus on those residues. Often a combination is needed – e.g., a pair of charge mutations + one aromatic mutation can convert a highly polyreactive antibody into an acceptable one. Importantly, after engineering, one must re-test the antibody in the polyreactivity assays to confirm improvement and ensure the antigen affinity is maintained.

Minimal Screening Panel Proposal

For early-stage developability screening (when material is limited), a minimal assay panel can effectively catch polyspecificity issues:

PSR (CHO-SMP) FACS or Bead assay: This is the primary screen for polyreactivity. Even with sub-milligram quantities, a flow cytometry based PSR or PSP assay can flag sticky antibodies. We use CHO membrane extract to represent a complex milieu and look for high binding outliers (e.g. antibodies whose MFI is above the 75th percentile of a reference set)
patents.google.com
.

PSP Multiplex Flow Assay: If available, the PSP assay (with, say, a mix of different reagents or beads) can be used in parallel. For instance, one could test binding to CHO lysate and to Ovalbumin in a bead-based multiplexed readout. This improves confidence, as true polyreactives will light up both. PSP is amenable to low amounts of IgG and is high-throughput
researchgate.net
.

BVP ELISA: A baculovirus particle (BVP) binding ELISA or similar assay can detect surface-stickiness (viruses present a negatively charged surface). It’s an inexpensive surrogate for cell surface binding. High BVP binding generally correlates with PSR
dspace.mit.edu
.

CIC (HPLC) or CSI (BLI): A Cross-interaction chromatography test (antibody as the mobile phase, interacting with a column of low pH proteins) or a self-interaction BLI can quantify self/self or self/other weak interactions. These highlight charge-driven self-association that often parallels polyspecificity. Moderate retention times are okay, but very long retention (or high self-interaction signals) are concerning.

HIC (Hydrophobic Interaction Chromatography): Even a small-scale HIC experiment (with a short analytical column) can reveal if an antibody has unusually high hydrophobicity. Compare the retention time to a panel of known clinical antibodies. If it’s an extreme outlier (retains much longer), that indicates a hydrophobic patch liability.

Using this mini-panel, one can accept or reject candidates based on cut-off criteria. For example: Accept an antibody if – PSR/PSP signal is below the 75th percentile of known good antibodies; CIC retention is not extreme (within normal range); HIC retention is moderate (not among the “sticky” outliers); and crucially, the antibody does not show induced polyreactivity after a low-pH challenge (meaning its PSR result doesn’t worsen after acid exposure). If an antibody passes all these, it’s likely developable. If it fails one or more, the data (which assays fail) will guide what to fix or whether to drop the candidate.

Modeling and Prediction Approach

Finally, an integrated in silico modeling recipe can be devised to predict polyspecificity from sequence and guide engineering:

Feature Computation: From the antibody sequence (Fv region), compute a vector of relevant features: {NetCDRCharge, AbsCDRCharge, V-region pI, Aromatic count, max positive patch score (from structure or homology model), HCDR3 length, Arg count in CDRs, sequence entropy features, etc.}. These represent the hypothesized contributors to polyreactivity.

Train Predictive Models: Using a training set of antibodies with known PSR/PSP results (and perhaps HIC results), train a regression or classification model. It may be useful to train separate models for different assay readouts – e.g., one model predicts the PSR MFI (which is largely charge-driven) and another predicts HIC retention (hydrophobic-driven). For PSR, features like charge and Arg count will weigh heavily; for HIC, aromatic and hydrophobic features will matter more. Indeed, one could use a multi-head neural network that shares the antibody representation but has two output heads (one for PSR-score, one for HIC-score), allowing it to learn a combined representation with bifurcating predictions.

Establish Risk Thresholds: The model’s output can be calibrated to known risk thresholds. For example, if historically an antibody with PSR MFI > some value or HIC retention > X was linked to clinical PK problems
pubmed.ncbi.nlm.nih.gov
pubmed.ncbi.nlm.nih.gov
, use that as a cutoff. The model can thus flag any new antibody whose predicted PSR or HIC score exceeds the 90th percentile of the acceptable range. In other words, a “polyspecificity risk index” can be derived – if the index is high, the antibody is likely to have developability issues and should be modified or deprioritized.

Iterate and Refine: Use the model predictions to suggest engineering changes (e.g., “the model says charge is too high – try mutating some Lys to reduce NetCharge”). After making sequence modifications in silico, re-run the features and predictions to see if the risk index improves. This way, one can computationally explore a panel of variants and select those predicted to be safer, before even making them.

This approach was exemplified by Chen et al. (2024), who demonstrated that using heavy-chain sequence properties and machine learning, one could predict an antibody’s propensity for nonspecific interactions with fairly high accuracy
pubmed.ncbi.nlm.nih.gov
. Over time, as more data is collected (especially from high-throughput platforms generating hundreds of data points
researchgate.net
researchgate.net
), these models will improve. The ultimate goal is to bake in developability considerations alongside binding affinity during antibody lead optimization – effectively designing out polyreactivity by using model feedback.

Key Takeaways

Polyreactivity (nonspecific, low-affinity binding to many molecules) is a critical liability for therapeutic antibodies, often driven by excess positive charge and hydrophobicity in the variable domains
pubmed.ncbi.nlm.nih.gov
. These features lead to “sticky” antibodies that bind cell membranes and various proteins nonspecifically.

The CHO lysate (PSR) binding assay is a powerful tool to detect polyreactivity, as CHO-derived soluble membrane proteins present a broad range of negatively charged and hydrophobic targets. Antibodies with high PSR or PSP signals usually have one or more of the following: positively charged CDR patches, aromatic/hydrophobic surface patches, and/or long flexible loops in the heavy chain
pubmed.ncbi.nlm.nih.gov
patents.google.com
.

Studies confirm that the heavy chain CDRs (especially HCDR3) dominate polyreactivity propensity
pubmed.ncbi.nlm.nih.gov
. Antibodies engineered to have balanced charge (no large net positive regions) and minimal exposed hydrophobes, particularly on the heavy chain, are far less likely to be polyspecific.

Entropy and flexibility contribute to polyspecificity: a more flexible antibody can adapt to bind multiple off-targets. External stresses like acidic pH can induce a polyreactive conformation
pubmed.ncbi.nlm.nih.gov
. CDR arginine mutations can also unexpectedly increase nonspecific binding
pubmed.ncbi.nlm.nih.gov
, so careful consideration must be given when making affinity/stability improvements.

Polyspecificity in vitro correlates with in vivo outcomes: High polyreactivity often means fast clearance and poor PK in animals and humans
pubmed.ncbi.nlm.nih.gov
. This is due to enhanced cellular uptake and altered FcRn recycling caused by charge/hydrophobic interactions
pubmed.ncbi.nlm.nih.gov
. Therefore, improving an antibody’s specific binding profile (and reducing off-target stickiness) is essential for a successful drug.

Mitigation strategies include sequence engineering (removing positive charges or hydrophobic residues, shortening loops, removing Arg hotspots) and process optimizations (avoiding low pH exposure). Even small changes like mutating a few residues can significantly drop PSR binding without harming target affinity, thus de-risking the antibody.

A combination of orthogonal assays (PSR/PSP, CIC, HIC, etc.) is recommended to fully characterize an antibody’s developability. Each assay provides insight into different interaction propensities, and together they ensure no major liability is overlooked.

Looking forward, computational approaches leveraging large datasets of antibody sequences and developability metrics are being developed to predict polyspecificity from sequence alone
pubmed.ncbi.nlm.nih.gov
. These models, coupled with high-throughput screening data, promise to guide the design of new antibodies that are both potent and specific, minimizing late-stage failures due to developability issues.

By paying close attention to CHO-PSR signals and related assays early, researchers can identify problematic antibodies (with charge/hydrophobic imbalances or excessive flexibility) and optimize them. Ultimately, polyspecificity is an avoidable pitfall – through informed design and screening, one can engineer therapeutic antibodies with high specificity, good stability, and low off-target interactions, thereby improving their chances of clinical success.

Sources:

Cunningham et al., 2021 – MAbs review on polyreactivity (figures on charge/hydrophobic effects)
pubmed.ncbi.nlm.nih.gov
pubmed.ncbi.nlm.nih.gov
.

US Patent US20160077105A1 (Adimab, 2016) – description of polyspecificity reagents (CHO & Sf9 lysates) and non-specific interaction mechanism
patents.google.com
patents.google.com
.

Makowski et al., 2021 – PSP assay development for antibody nonspecific binding (flow cytometry method)
researchgate.net
researchgate.net
.

Chen et al., 2024 – large-scale analysis linking heavy chain features to polyreactivity (Cell Reports)
pubmed.ncbi.nlm.nih.gov
.

Boughter et al., 2020 – eLife study on sequence determinants of polyreactivity (information theory approach)
pubmed.ncbi.nlm.nih.gov
pubmed.ncbi.nlm.nih.gov
.

Arakawa & Akuta, 2023 – review on mechanistic insight into polyreactivity, highlighting acid and arginine effects
pubmed.ncbi.nlm.nih.gov
pubmed.ncbi.nlm.nih.gov
.

Avery et al., 2018 – Pfizer study correlating in vitro polyspecificity assays with in vivo clearance
pubmed.ncbi.nlm.nih.gov
pubmed.ncbi.nlm.nih.gov
.