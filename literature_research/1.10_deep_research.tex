\documentclass[12pt]{article}
\usepackage[utf8]{inputenc}
\usepackage[T1]{fontenc}
\usepackage{authblk}
\usepackage{geometry}
\usepackage{hyperref}
\geometry{margin=1in}

\title{\textbf{Hydrophobic Interaction Chromatography as a Developability Readout in the GDPa1 Antibody Dataset}}
\author{}
\date{}

\begin{document}
\maketitle

\begin{abstract}
Hydrophobic interaction chromatography (HIC) is a key biophysical assay in the GDPa1 dataset of therapeutic antibodies, serving as a proxy for surface hydrophobicity. This report examines how HIC retention behavior reflects various biophysical liabilities in GDPa1, including aggregation propensity, self-association, high viscosity, and clearance risk. We detail what HIC measures and why it correlates with developability issues, then review modeling approaches used to predict HIC outcomes in GDPa1: from sequence-based heuristics (e.g. CDR hydrophobicity and charge patches) to structure-based descriptors (e.g. exposed hydrophobic patch area, solvent-accessible surface, packing density) and modern regression or machine learning models. We also discuss the experimental setup and limitations of HIC as an assay, such as salt dependence and confounding factors, and we outline protein engineering strategies to mitigate hydrophobic liabilities flagged by HIC. Finally, we highlight how HIC predictions and measurements link to other developability readouts in GDPa1 (like AC-SINS self-association, polyreactivity assays, and solubility metrics), emphasizing a multi-factor approach to antibody developability assessment. All statements are supported by peer-reviewed literature or preprints, with dense citations for transparency.
\end{abstract}

\section{Introduction}
Antibodies chosen for therapeutic development must not only bind their targets with high affinity but also exhibit favorable “developability” properties: high stability, solubility, and expression, along with low aggregation propensity, viscosity, polyreactivity, and clearance risk
ginkgo.bio
. Deficiencies in these properties can lead to manufacturing difficulties or clinical failure
ginkgo.bio
. To enable early identification of such liabilities, the GDPa1 dataset was recently released, containing paired sequences and experimental developability data for 242–246 clinical-stage IgG antibodies across 9–10 assays
huggingface.co
huggingface.co
. One of these assays is \textbf{Hydrophobic Interaction Chromatography (HIC)}, which measures antibody hydrophobicity. HIC has emerged as a critical screen for antibodies, since excessive surface hydrophobicity is linked to numerous biophysical liabilities
medium.com
. In fact, retrospective analyses of clinical antibody datasets have shown that in vitro “flags” for high hydrophobicity or polyspecificity correlate with poorer clinical progression, underscoring their predictive value
pubmed.ncbi.nlm.nih.gov
.

In this report, we focus on how HIC data from GDPa1 reflect underlying biophysical issues and how one can model and mitigate these issues. We first explain what HIC retention time signifies at the molecular level and why it is an important developability indicator. We then explore modeling approaches for HIC in GDPa1, including sequence-based predictors (e.g. CDR loop hydrophobicity metrics and charged patch calculations) and structure-based or data-driven models (e.g. hydrophobic surface patch analysis, solvent-accessible surface areas, and machine-learning regressors). Next, we discuss the experimental setup of HIC in a high-throughput context and the limitations or confounding factors of this assay (such as resolution limits and salt conditions). We then outline protein engineering strategies to alleviate hydrophobicity-related liabilities identified by HIC (for example, mutational approaches to reduce exposed hydrophobic patches). Finally, we connect HIC with other developability readouts in GDPa1—like self-association by AC-SINS, polyspecificity in a CHO binding assay, and solubility measures—to illustrate how HIC complements these assays in a holistic developability profile. All claims are supported with citations from the recent literature or preprints to ensure verifiability.

\section{HIC as a Measure of Antibody Hydrophobicity and Liabilities}
HIC is a chromatographic technique that separates proteins based on hydrophobic interactions with the stationary phase. Antibodies are typically loaded onto a mildly hydrophobic column in high salt (which promotes hydrophobic binding), then eluted with a decreasing salt gradient. Antibodies with more exposed hydrophobic surfaces bind the column more strongly and thus elute later (longer retention time)
medium.com
. In practice, the HIC retention time of an IgG is used as a quantitative proxy for its overall surface hydrophobicity
academic.oup.com
. GDPa1 defines “hydrophobicity by HIC” as one of the core developability metrics
huggingface.co
, treating a long HIC retention (beyond a threshold) as a red flag for developability. Indeed, a HIC retention above a certain cutoff (set based on the distribution for approved mAbs) is considered “excessive hydrophobicity” and tends to correlate with unfavorable behaviors like aggregation and polyreactivity
medium.com
.

Mechanistically, a high HIC retention indicates that the antibody has significant hydrophobic patches on its surface. Such patches can mediate non-specific intermolecular interactions. For example, hydrophobic regions on different antibody molecules can associate, leading to \textbf{aggregation} (especially under stress or concentration)
academic.oup.com
. It is well documented that partially unfolded or unstable antibodies expose buried hydrophobic residues, which then nucleate irreversible aggregates
academic.oup.com
. Even in fully folded antibodies, large solvent-exposed hydrophobic patches (often located in complementarity-determining regions, CDRs) are associated with higher aggregation propensity
tandfonline.com
. Thus, HIC is indirectly a readout of aggregation risk. Consistently, prior studies have found that antibodies known to aggregate often exhibit long HIC retention times
medium.com
.

Hydrophobic surface patches can also cause \textbf{self-association} in solution, even without forming large aggregates. Antibodies with hydrophobic or “sticky” patches may weakly self-interact or form reversible oligomers, which is problematic at the high concentrations needed for injection. This manifests as high solution viscosity and poor colloidal stability. Indeed, HIC is often used alongside self-interaction assays to flag such issues; molecules with high HIC hydrophobicity frequently show strong self-association in assays like AC-SINS or DLS $k_d$, leading to high viscosity
medium.com
. In GDPa1, the HIC measurement was found to strongly correlate with an independent colloidal stability metric, the SMAC assay (Spearman $\rho\approx0.78$)
biorxiv.org
. SMAC (standup monolayer adsorption chromatography) retention is inversely related to colloidal stability
researchgate.net
, so antibodies that bind strongly in SMAC (poor colloidal stability) tend to also have high HIC retention. This clustering of HIC with colloidal stability indicates that hydrophobicity-driven self-interaction is a major common factor
biorxiv.org
. More generally, hydrophobic patches on an antibody surface can lead to non-specific self-association in solution
academic.oup.com
, which in turn can cause high viscosity at formulation concentrations. Thus, HIC is an early indicator for potential viscosity issues: antibodies with the longest HIC retentions are often those that would be challenging to formulate at high concentration due to self-association.

Another liability linked to high HIC hydrophobicity is \textbf{polyreactivity} or polyspecificity, which is the tendency of an antibody to bind off-target antigens or surfaces (such as membrane proteins, nucleic acids, or other proteins) non-specifically. Antibodies with excessive hydrophobic or aromatic content in their antigen-binding site can promiscuously stick to unrelated molecules or exposed hydrophobic regions on proteins
pmc.ncbi.nlm.nih.gov
. Empirically, polyreactive antibodies often show high HIC retention, as both properties stem from similar physicochemical features (exposed hydrophobic patches)
medium.com
. In the GDPa1 panel, polyreactivity was measured by binding to Chinese hamster ovary (CHO) cell membrane extracts and to ovalbumin; those two polyreactivity readouts strongly correlate with each other
medium.com
, and antibodies scoring high in these polyspecificity assays tend to also be among those with high HIC hydrophobicity. In a large-scale analysis of therapeutic mAbs, in vitro hydrophobicity and polyreactivity assays together were the best predictors of an antibody’s likelihood to progress in development
pubmed.ncbi.nlm.nih.gov
. This underlines that an antibody with a high HIC retention (hydrophobic surface) is more likely to exhibit off-target binding or promiscuous interactions (thus getting flagged in polyreactivity assays) and may face greater risk of failure.

Finally, surface hydrophobicity can affect an antibody’s \textbf{pharmacokinetics} and clearance in vivo. Antibodies with “sticky” surfaces may interact non-specifically with serum proteins or cell surfaces, leading to faster clearance from circulation
pmc.ncbi.nlm.nih.gov
. They might also be recognized by clearance receptors or tend to form sub-visible aggregates that get cleared by the immune system. It has been observed that both high polyreactivity and high hydrophobicity contribute to reduced half-life and bioavailability of antibodies in vivo
pmc.ncbi.nlm.nih.gov
. For instance, an antibody that binds strongly to heparin (indicative of a highly cationic patch) or has a very long HIC retention (indicative of a hydrophobic patch) can have increased non-specific tissue uptake or clearance
medium.com
. Many developability guidelines therefore include HIC (hydrophobicity) and heparin binding (polycationicity) thresholds as surrogates for potential clearance issues
medium.com
. In GDPa1, a heparin affinity chromatography (HAC) assay was included to gauge polycationic patch liabilities
huggingface.co
, and the authors note a trade-off: antibodies with extremely high positive charge (high heparin binding) often have lower hydrophobicity, whereas those with high hydrophobicity may not bind heparin as much
pmc.ncbi.nlm.nih.gov
. Both extremes can cause clearance problems, but HIC focuses on the hydrophobic side of that spectrum. In fact, combining HIC and self-association metrics has been proposed as a way to predict in vivo pharmacokinetic behavior: antibodies with both high HIC retention and strong self-interaction signals are the most likely to have poor PK profiles
wuxibiologics.com
.

In summary, HIC retention time serves as a convenient “umbrella” readout for multiple developability liabilities. A high HIC value flags an antibody as having a hydrophobic surface, which is directly linked to higher risk of aggregation, self-association (and hence high viscosity), non-specific/polyreactive binding, and faster clearance
medium.com
pmc.ncbi.nlm.nih.gov
. This is why HIC is emphasized in GDPa1 and similar datasets. In contrast, antibodies with low HIC retention (indicating more hydrophilic surfaces) tend to be easier to formulate and less prone to the aforementioned liabilities. Notably, no single assay can capture every aspect of developability
scispace.com
, but HIC provides a central piece of the puzzle by targeting one of the most important molecular properties underlying many failure modes: surface hydrophobicity.

\section{Modeling HIC in GDPa1: Sequence-Based and Structure-Based Approaches}
One goal of GDPa1 is to enable predictive modeling of developability from antibody sequence or structure
openreview.net
openreview.net
. Here we review how HIC (hydrophobicity) can be modeled or predicted using various approaches, as evidenced by literature and the GDPa1 initiative. We cover three categories: (1) sequence-based heuristic predictors that derive hydrophobicity indices or features directly from the amino acid sequence (including CDR properties and charge/hydrophobic patch motifs), (2) structure-based modeling that uses 3D antibody models or descriptors (such as exposed hydrophobic surface area, patches, and packing), and (3) data-driven regression or QSAR-style models, often high-throughput machine learning, that learn to predict HIC values from either sequence or structure descriptors.

\subsection{Sequence-Based Predictors (CDR Hydrophobicity and Charge Patches)}
Even without an explicit structure, certain sequence features are known to correlate with HIC retention. Because antibodies are modular, one can often attribute developability-relevant properties to specific sequence regions (e.g. CDR loops or framework segments). A straightforward approach is to calculate the overall hydrophobicity of the CDRs using scales like Kyte-Doolittle or Gibbs transfer energies. For example, an antibody’s “CDR hydrophobicity score” can be computed by summing the hydrophobic indices of all residues in the CDR loops (perhaps normalized by length). Prior studies have shown that antibodies with unusually hydrophobic CDR sequences tend to exhibit poor developability. In one case, the antibody Galiximab was noted to have a large hydrophobic patch in its CDR-H3 loop; a computed “PSH” (patch surface hydrophobicity) score in the CDR vicinity was extremely high for this antibody
pmc.ncbi.nlm.nih.gov
, correlating with its known aggregation and HIC behavior. In general, the aggregation propensity of mAbs has been found to associate with the hydrophobicity of their CDRs
tandfonline.com
. Thus, simple sequence metrics like “% hydrophobic residues in CDR3” or “average hydropathy of CDR loops” can serve as predictors for HIC rank: higher values often imply longer HIC retention.

Beyond average hydrophobicity, specific \textbf{patterns or patches} in the sequence are important. For instance, a contiguous stretch of hydrophobic amino acids in a CDR (especially CDR-H3) might indicate a surface patch when folded. Certain motifs such as multiple consecutive tyrosine, tryptophan, or valine residues in CDRs have been linked to non-specific binding. Some computational tools identify “hydrophobic patches” directly from sequence by sliding a window or using known antibody liabilities databases. Grinshpun \textit{et al}. and others have defined sequence-based patch scores that highlight clusters of hydrophobic or aromatic residues in the variable region
pubmed.ncbi.nlm.nih.gov
. These sequence patches often correspond to the regions that drive HIC retention. Modern developability guidelines advise checking for such motifs early in discovery: e.g. a CDR containing a “YYWG” stretch or a long run of non-polar residues would raise concern for high HIC and aggregation.

Another important sequence feature is the distribution of \textbf{charged residues}, which affects both solubility and how the antibody behaves in HIC (since HIC is done in high salt, charges are partly shielded, but extreme charge patterns still matter). For example, large clusters of positively charged residues (lysine, arginine) on one face of the antibody can cause polyreactivity (binding to negatively charged molecules like DNA or heparin) and might also indirectly influence HIC by altering how the protein sits on the hydrophobic surface
tandfonline.com
. While HIC primarily measures hydrophobic interactions, antibodies with very high net positive charge sometimes show shorter-than-expected retention due to repulsive effects or altered conformation in high-salt conditions
tandfonline.com
. Conversely, a high net negative charge can sometimes increase retention by reducing aggregation during the assay and keeping the protein more soluble until elution. Therefore, sequence-based predictors often include features like overall isoelectric point (pI) or the presence of charged patches. In silico metrics like “positive charge patch score” and “negative patch score” can be computed from sequence (e.g. counting the number of basic residues in CDRs or the largest stretch of them). These complement the hydrophobic patch metrics: for instance, a developability index (DI) introduced by researchers combines a hydrophobicity-based score (such as SAP, see below) with the antibody’s net charge
frontiersin.org
. The rationale is that an antibody with a high hydrophobic score but also a high net charge might behave slightly better than hydrophobicity alone would suggest (because charge repulsion can counteract aggregation), whereas high hydrophobicity coupled with neutral charge is especially risky
frontiersin.org
. Sequence-derived charge features help refine HIC predictions in borderline cases.

Several published models demonstrate sequence-only prediction of HIC. Jain \textit{et al}. (2017) devised a machine learning approach to predict “delayed” HIC retention directly from sequence, without actual structures
academic.oup.com
academic.oup.com
. They trained a random forest model on known antibody sequences and HIC measurements. A key aspect was to estimate each residue’s likely solvent exposure from sequence context: the model learned to predict which parts of the sequence are surface-exposed hydrophobics versus buried, essentially inferring a pseudo-structure
academic.oup.com
. Using such features, they achieved good accuracy in classifying antibodies as having high or normal HIC retention (ROC AUC $\approx0.85$ in cross-validation)
academic.oup.com
. The success of that approach highlights that much of the HIC signal is encodable in sequence features—particularly the presence of exposed hydrophobic residues in the variable domain. Notably, their method derived an “amino acid hydrophobic propensity scale” optimized for HIC prediction
academic.oup.com
, essentially assigning weights to each amino acid for how much it contributes to HIC when exposed. This can be seen as a refined sequence-based hydrophobicity index tailored to antibodies. The GDPa1 dataset enables similar sequence-based modeling. In fact, simple linear models using pre-trained protein language model embeddings (like ESM-2) have already been applied to GDPa1: a ridge regression on ESM-2 sequence embeddings achieved a Spearman $\rho\approx0.42$ for predicting HIC values across the 246 antibodies
openreview.net
. This performance, while modest, improved with dataset size
medium.com
, and it confirms that the sequence contains significant information about HIC. Features like CDR hydropathy and patchiness likely underlie what the embeddings and models detect. In summary, sequence-based predictors—ranging from simple heuristic scores (CDR hydrophobicity, charge clusters) to machine learning on sequence features—constitute an effective and high-throughput way to model HIC in the absence of structure, and they form a baseline for more complex modeling efforts.

\subsection{Structure-Based Modeling (Hydrophobic Patches, SASA, and Packing)}
When an antibody’s 3D structure (experimental or modelled) is available, one can directly compute biophysical descriptors that relate to HIC. Structure-based modeling is especially pertinent given that GDPa1 has provided predicted Fab structures for all antibodies (using ABodyBuilder3)
huggingface.co
. Key structure-derived features include: the size and location of hydrophobic surface patches, the total solvent-accessible non-polar surface area, the tightness of packing (which relates to stability), and the distribution of charged vs hydrophobic surface regions.

A primary descriptor is the \textbf{largest hydrophobic patch} on the antibody’s surface. Various definitions exist, but a common one is the largest cluster of surface atoms or residues with high hydrophobicity that are contiguous (within some distance cutoff) on the solvent-accessible surface. Tools like MOE (Molecular Operating Environment) and others (e.g. the “SAP” or “HPatch” metrics) can quantify this. Park and Izadi (2024) introduced a set of molecular surface descriptors for exactly this purpose
pubmed.ncbi.nlm.nih.gov
. They compute properties such as HPATCH (hydrophobic patch size by different scales) and showed that these correlate with experimental developability data including HIC
pubmed.ncbi.nlm.nih.gov
. In their benchmarking, a descriptor focusing on CDR surface hydrophobicity (CDR_HPATCH using the Black-Mould scale) had one of the strongest correlations with HIC retention times for a panel of 137 clinical antibodies
pubmed.ncbi.nlm.nih.gov
. Specifically, they found that the Spearman correlation of this hydrophobic patch measure with HIC data (from Jain \textit{et al}. 2017) was significant, and it performed well in classifying antibodies with high HIC (above a retention threshold of 11.7 min)
pubmed.ncbi.nlm.nih.gov
. This indicates that simply knowing the size of the largest hydrophobic patch on the Fab surface can largely explain whether an antibody will have unusually high hydrophobic interaction. Other methods like Spatial Aggregation Propensity (SAP) similarly estimate a hydrophobic patch score from structure by coloring the antibody surface according to hydrophobicity and finding hotspots
pubmed.ncbi.nlm.nih.gov
. SAP was originally developed to predict aggregation-prone regions in proteins; for antibodies, SAP has been used as an in silico predictor of aggregation and non-specific binding propensity. Studies have combined SAP with charge to create a Developability Index, which was shown to correlate with experimental HIC and self-interaction outcomes
frontiersin.org
. Thus, structure-based hydrophobic patch calculations are very direct predictors of HIC: a large patch (especially if on a solvent-exposed area of the Fv) will likely yield a longer retention time
pubmed.ncbi.nlm.nih.gov
, whereas antibodies with very dispersed or minimal hydrophobic surface area tend to elute earlier in HIC.

Another set of structure features involve \textbf{solvent accessible surface area (SASA)}, particularly the non-polar component. One can compute the total SASA of hydrophobic side chains on the antibody. A higher exposed hydrophobic SASA generally correlates with higher HIC retention, since more hydrophobic surface is available to interact with the HIC column. Earlier HIC studies showed that retention times correlate with amino-acid propensities weighted by surface area obtained from 3D structures
academic.oup.com
. In other words, if one takes a solved Fab structure and sums contributions of each residue (for example, using a coefficient for each amino acid type times the residue’s solvent-exposed area), the resulting score correlates with HIC
academic.oup.com
. This approach effectively produces a “hydrophobic exposure index.” The 2017 Jain model took a step towards this by predicting surface exposure from sequence, as mentioned. With actual structures, one can do it explicitly. In GDPa1’s context, using predicted Fab structures from ABodyBuilder, one could calculate each antibody’s exposed hydrophobic area or a related index and use that in regression models for HIC. These structure-based models often outperform purely sequence models because they account for conformational effects (e.g. a hydrophobic residue might be buried in one antibody but exposed in another due to structural context). Park and Izadi’s work, for example, examined how different structure prediction methods can lead to shifts in surface descriptor values
pubmed.ncbi.nlm.nih.gov
pubmed.ncbi.nlm.nih.gov
, underscoring that having an accurate structure is important. They also noted that averaging descriptors over multiple conformations (via molecular dynamics sampling) improved consistency
pubmed.ncbi.nlm.nih.gov
, indicating that slight structural differences can affect patch measurements. Nonetheless, even an approximate structure is usually sufficient to identify major hydrophobic patches (like a large aromatic cluster on a CDR loop).

\textbf{Charge and polar patches} can likewise be mapped on the structure. While HIC is less sensitive to charge (due to high salt conditions), extremely uneven charge distribution can modulate effective hydrophobicity. For instance, an antibody might have a big hydrophobic area that is partially shielded by an adjacent positively charged region; under high salt, the charge shielding is reduced, so the hydrophobic area might behave as if larger. Conversely, if a hydrophobic patch is surrounded by charged residues, in low-salt conditions that antibody might aggregate (charges not shielding) but in high-salt (HIC) it becomes very sticky. These nuanced effects mean structure models can help by revealing the spatial arrangement of hydrophobic vs. charged residues. Tools like APBS can compute electrostatic surface maps
pubmed.ncbi.nlm.nih.gov
, and one can identify clusters of like charge on the surface. A “positive surface patch” descriptor (e.g. the largest cluster of contiguous basic residues on the surface) might correlate with, say, heparin binding or polyreactivity, whereas an “electrostatic dipole” moment might correlate with nonspecific binding. While these may not directly correlate with HIC retention strongly, they are part of multi-variate models. Indeed, Park and Izadi assessed descriptors like CDR_APBS_neg (negative electrostatic potential in CDR) and found some correlate with viscosity or other properties
pubmed.ncbi.nlm.nih.gov
, though not all with HIC. They proposed a set of six in silico flags combining hydrophobic and electrostatic surface features
pubmed.ncbi.nlm.nih.gov
 to cover different liabilities. In structure-based HIC modeling, typically one or two principal components emerge: one corresponding to hydrophobic surface area (dominant for HIC), and another related to something like charge or polarity which might slightly tweak the prediction.

\textbf{Packing density} and conformational stability factors can also play an indirect role. An antibody with poor internal packing or low stability might partially unfold or fluctuate, exposing hydrophobic regions transiently. This could lead to anomalously high HIC retention or broad peaks if the antibody partially unfolds on the column. While standard HIC analysis doesn’t explicitly account for unfolding, extremely unstable antibodies might behave inconsistently. No direct descriptor for “packing” is commonly used in HIC prediction, but one could use the fraction of buried hydrophobic residues or core packing metrics from the structure as a proxy. If an antibody has a loosely packed core, it’s a candidate for unfolding-mediated aggregation under stress
academic.oup.com
. However, in GDPa1 (which is based on well-behaved clinical antibodies), most antibodies are likely properly folded and stable during the assay. So packing density might correlate more with thermal stability (Tm) than with HIC. Still, including a stability term in predictive models can sometimes improve them. For example, one might combine a hydrophobic surface score with the antibody’s melting temperature (if known) to predict aggregation propensity in a more holistic way
sciencedirect.com
, though HIC alone is directly hydrophobicity-focused.

In summary, structure-based modeling of HIC leverages detailed spatial information: measuring how much hydrophobic surface the antibody presents and in what configuration. The largest hydrophobic patch and total hydrophobic SASA are strong predictors of retention
pubmed.ncbi.nlm.nih.gov
academic.oup.com
. Studies have validated that purely structure-derived hydrophobic indices correlate well with experimental HIC data (often with $\rho$ or $R^2$ in the 0.4–0.6 range for diverse sets)
pubmed.ncbi.nlm.nih.gov
. With GDPa1’s structural annotations, one can compute these descriptors for each antibody and train predictive models. Indeed, combining multiple structural features can yield a model that not only predicts HIC but is also interpretable (e.g. showing which patch drives the score). This approach is akin to QSAR (Quantitative Structure-Activity Relationships) in small-molecule science, except here it’s structure-property relationship for proteins. We next discuss such data-driven models in more detail.

\subsection{QSAR-Style and Machine Learning Models for HIC}
QSAR-style models treat the HIC retention (a continuous value or rank) as a function of various molecular descriptors, which can be sequence- or structure-derived. The difference from the above heuristics is that in a QSAR or ML model, the weights and combination of features are optimized automatically to best fit the data, rather than predefined by human insight. Given the relatively large size of GDPa1 (242 antibodies with HIC and other labels)
huggingface.co
, modern machine learning methods can be applied to learn predictive models for HIC.

One simple yet effective approach, as demonstrated by Arsiwala \textit{et al}. (2025), is to use pretrained protein language model embeddings as features
openreview.net
. In their work, each heavy and light chain sequence was passed through a transformer model (ESM-2), and the resulting vector embeddings were fed into a ridge regression to predict each developability property
openreview.net
openreview.net
. For HIC, this model achieved a Pearson $R\approx0.34$ and Spearman $\rho\approx0.42$ on a clustered cross-validation
openreview.net
. This indicates the model captured some signal: in effect, the embedding likely encodes motifs like hydrophobic stretches or unusual residue compositions that correlate with HIC. Interestingly, this performance was obtained with only ~246 training points; it suggests that with more data, even better results are possible
medium.com
. The advantage of such ML models is that they consider complex nonlinear combinations of sequence features. For example, an embedding-based model might implicitly detect that a tryptophan in a certain position of CDR-H2 combined with a glycine deletion in CDR-L3 leads to a sticky patch. These patterns could be too subtle for a simple human-defined rule but can be learned from data.

Another QSAR approach is to use a broad set of physicochemical descriptors (both sequence and structure) as inputs to a regression. For instance, one could calculate ~20 features for each antibody (e.g. % hydrophobic residues, CDR3 length, net charge, predicted aggregation “AggScore”, SAP score, etc.) and then use a stepwise regression or machine learning (random forest, support vector machine, gradient boosting) to map those to the measured HIC retention. Tushar Jain and colleagues in 2023 evaluated many published in silico metrics on a collection of antibodies
pubmed.ncbi.nlm.nih.gov
. They found that models often overfit to their training sets, and generalization is challenging
pubmed.ncbi.nlm.nih.gov
. This underscores that a QSAR model must be trained carefully (preferably with cross-validation on diverse splits, as GDPa1 suggests with hierarchical cluster folds
huggingface.co
). In their review, they note that in-silico predictors of hydrophobicity (like certain patch calculations) did correlate with experimental outcomes, but using them in isolation was not as predictive of clinical success as the actual in vitro assays
pubmed.ncbi.nlm.nih.gov
. That said, combining multiple features can improve robustness. For HIC specifically, a model might combine a sequence-based feature (e.g. hydrophobic motif count) with a structure feature (e.g. SAP score) and perhaps a term for glycosylation or other known modifiers.

The 2017 Bioinformatics paper by Jain \textit{et al}. is an example of a targeted QSAR model: they estimated surface exposure from sequence and derived an amino acid propensity scale fitted to HIC data
academic.oup.com
academic.oup.com
. In effect, they built a linear model where HIC retention is the sum of contributions from each residue (with those contributions modulated by whether the residue is likely exposed). The fact this worked well (AUC 0.85 for identifying high-retention antibodies
academic.oup.com
) suggests that, at least for identifying outliers, a relatively linear model in meaningful features is sufficient. Random forest was used likely to capture some nonlinear interactions, but even a logistic regression might have done well given the right features.

In a high-throughput industrial context, one can imagine a pipeline where tens of thousands of antibody sequences are passed through such a model to prioritize which ones are likely to have acceptable HIC profiles. Because the GDPa1 platform was built to generate “ML-ready” data
ginkgo.bio
, it supports training more advanced models too. For instance, one could fine-tune a deep learning model (like a small transformer or LSTM) directly on the sequence-to-HIC mapping. With 242 data points, a deep model would overfit, but as data grows (the platform can produce thousands of data points per week
ginkgo.bio
), this may become viable. There has also been interest in multi-task models that predict several developability metrics at once, leveraging the fact that some assays correlate. A multi-output model could predict HIC, AC-SINS, and polyreactivity simultaneously, potentially improving its internal features by using the shared structure between these properties.

As an analog to classical QSAR in small molecules (where descriptors like logP, polar surface area, etc., feed a regression for, say, solubility), here the descriptors might be things like: \emph{HPatch size, SAP score, net charge, Fv isoelectric point, number of buried polar groups, etc.} A regression model (say partial least squares or an ensemble tree model) could be trained on GDPa1 data to predict HIC retention times. Because HIC is a quantitative value (in minutes or column volumes), one can use standard regression metrics (MSE, $R^2$). If treated as classification (flag high vs normal), one can optimize AUC or similar. Notably, Park and Izadi (2024) presented precision-recall curves and $R^2$ for various descriptors predicting HIC above a threshold
pubmed.ncbi.nlm.nih.gov
, showing that certain features (like MOE’s CDR_HYD or TAP’s CDR.PSH) had slightly better performance than others
pmc.ncbi.nlm.nih.gov
. In their results, a classical descriptor “CDR_HYD” (presumably average CDR hydrophobicity) marginally outperformed others for binary classification of high HIC cases
pmc.ncbi.nlm.nih.gov
. This reinforces that relatively simple features can be powerful, but combining them in a model might capture more antibodies that are on the cusp.

In summary, QSAR-style and ML modeling for HIC in GDPa1 spans from interpretable linear models using a few intuitive features to black-box models using high-dimensional sequence embeddings. The unifying theme is that they all attempt to predict the same outcome that the HIC experiment measures: how “sticky” the antibody is under hydrophobic interaction conditions. The GDPa1 dataset allows these models to be validated robustly (with held-out sequences and cluster splits)
huggingface.co
. Early results (ridge on ESM embeddings, random forest on sequence features) have shown moderate success
openreview.net
academic.oup.com
. As data volume grows (e.g. incorporating more antibodies or replicate measurements), one can expect the predictive performance to improve. Ultimately, the aim is to use such models in antibody design — for example, generative models guided by a HIC predictor
openreview.net
openreview.net
 to propose mutations that lower hydrophobicity while maintaining affinity. The feasibility of that was recently demonstrated by Arsiwala \textit{et al}., who guided a sequence generator with a combined HIC and AC-SINS predictor to design novel antibodies with improved predicted developability
openreview.net
. This synergy between high-throughput assays like HIC and modern ML models heralds a future where one can computationally screen or even design-out hydrophobic liabilities before ever needing wet lab confirmation.

\section{HIC Experimental Setup and Assay Limitations}
While HIC is a valuable assay, it is important to understand its practical execution and limitations, especially in the context of a high-throughput platform like GDPa1’s PROPHET-Ab
ginkgo.bio
. The experimental setup for HIC in GDPa1 involves running each antibody on a hydrophobic interaction chromatography column under standardized conditions. Typically, antibodies were formulated in a high-salt buffer (e.g. with ammonium sulfate) and loaded onto a HIC column (often butyl or phenyl-functionalized resin). A gradient decreasing the salt concentration elutes the antibodies in order of increasing hydrophobicity. The output is a retention time (or volume) for the main elution peak of each IgG. In GDPa1, these retention times were measured in triplicate and median-averaged for robustness
huggingface.co
. The resulting “HIC score” in the dataset is effectively the normalized retention time for the antibody’s monomeric peak.

One of the achievements of PROPHET-Ab was to scale HIC to high throughput without losing reproducibility. They reported that their automated HIC measurements were highly consistent with legacy low-throughput methods
ginkgo.bio
. In fact, HIC retention times from the GDPa1 platform showed a near one-to-one correlation with those reported for the same antibodies in an earlier study (Jain et al. 2017), with Spearman $\rho=0.97$ over 133 overlapping antibodies
biorxiv.org
. This indicates excellent reproducibility and suggests that the HIC method itself (column type, buffer conditions) was comparable to that earlier work. Nonetheless, even with high reproducibility, HIC has several inherent limitations and potential confounders:

1. Resolution and Dynamic Range: HIC can distinguish antibodies with substantially different hydrophobicity, but very fine differences may be hard to resolve. For example, if two antibodies elute at 12.0 min and 12.3 min, is that difference meaningful or within run-to-run variability? The GDPa1 authors likely determined a threshold above which retention is considered problematic (perhaps analogous to the 90th percentile of approved mAbs, as Engin points out
medium.com
). Values beyond that threshold (e.g. >13–14 min on their setup) might be “delayed retention.” However, values near each other might not be statistically separable. In high-throughput mode, slight shifts in retention could occur due to instrument variability or gradient differences. Thus, HIC is often treated as a relative ranking rather than an absolute measure. Also, extremely hydrophobic antibodies might elute only at the very end of the gradient or require an organic solvent wash to come off the column. In such cases, their “retention time” may just be recorded as a cutoff (e.g. “$>$X min”). This censoring can limit the quantitative range. While GDPa1 hasn’t explicitly stated if any antibodies stuck to the column, such behavior is possible for very hydrophobic or aggregation-prone ones and represents a limitation (those samples might yield broad peaks or tailing in the chromatogram rather than a sharp peak).

2. Salt and buffer conditions: The HIC outcome can depend on the salt type, concentration, and gradient. High salt is needed to enhance hydrophobic binding; GDPa1 likely used a specific molarity of ammonium sulfate or similar, chosen to give good separation for their antibody set. However, if one changed the salt concentration, the absolute retention times would shift. This means HIC data from different labs are not directly comparable unless conditions are matched or calibrated
biorxiv.org
. The Jain 2017 dataset and GDPa1 dataset aligning well
biorxiv.org
 suggests they used similar conditions. Another factor is pH: typically HIC is done around neutral pH, but if pH were changed, the protein’s charge state changes and could alter retention (though less than in ion-exchange since hydrophobic interactions dominate in high salt). The gradient shape (linear vs step) and flow rate can also affect resolution. In high-throughput, there is pressure to shorten run times. If runs are too short (fast gradients), resolution between similar antibodies might suffer. GDPa1 likely optimized the gradient to balance throughput and resolution.

3. Confounding by Charge and Other Interactions: Although HIC is designed to probe hydrophobic interactions, antibodies are complex molecules and other interactions can come into play. One known confounder is the presence of extremely high positive charge on the antibody. HIC resins often carry a weak negative charge (depending on ligand and support), and at high salt most electrostatic interactions are suppressed. However, if an antibody has a strongly cationic patch (e.g. many lysines in one region), it might experience a mild electrostatic repulsion or attraction that alters its effective retention. Teroerde et al. (2025) observed that differences in HIC for multispecific formats reflected not only hydrophobicity changes but also charge distribution changes
tandfonline.com
. This suggests that engineering a molecule can simultaneously affect charge and hydrophobicity, and HIC retention is a composite outcome. For example, if one engineered an antibody to reduce hydrophobicity but in doing so introduced several extra positive charges, one might find the HIC retention did not drop as much as expected or even paradoxically increased due to reduced solubility or different binding mode
tandfonline.com
. Another confounder is glycosylation: if an antibody has an unmodeled glycosylation (e.g. in the variable domain, which can happen in some engineered or mouse antibodies), the carbohydrate is very hydrophilic and could reduce HIC retention dramatically. GDPa1’s antibodies are IgGs, mostly with the usual Fc N-glycan, which is constant; variable domain glycans are rarer but if present, they would make an antibody elute earlier (appear less hydrophobic) than its sequence alone might suggest. Conversely, removing a glycan (say by mutation of an NXT motif in the variable region) can increase hydrophobic exposure and HIC retention. One study noted that eliminating a Fab glycosylation led to higher polyreactivity, and to compensate, nearby aromatic residues had to be mutated to reduce the patch’s hydrophobicity
europepmc.org
. This highlights how post-translational features can skew HIC results.

HIC is also sensitive to the folded state of the protein. If an antibody sample has some fraction of misfolded or partially unfolded species, those species might either (a) precipitate on the column (and not elute) or (b) elute separately, causing shoulder peaks or broadening. Heterogeneity in HIC can thus indicate conformational issues. In GDPa1, they primarily report a single value per antibody, presumably for the main monomer peak
huggingface.co
. However, one should keep in mind that if an antibody had notable aggregate or fragment content, that would usually be screened out by SEC or purity assays (which GDPa1 also includes: SEC for aggregation, CE-SDS for purity
huggingface.co
). So HIC usually was applied to fairly pure monomer samples, focusing on conformationally intact IgGs. Still, minor issues like proline cis-trans isomers or local unfolding might subtly affect retention.

4. Throughput and reproducibility considerations: Running hundreds of antibodies on HIC means that slight day-to-day variations (column aging, buffer prep differences) could introduce noise. GDPa1 mitigated this by median-averaging replicates
huggingface.co
 and likely randomizing sample order. However, one limitation is that comparisons across different experimental batches may require normalization. They likely included some control antibodies run repeatedly to calibrate across runs. Without such calibration, one run’s retention times could be offset by e.g. 0.2 min from another’s. In the dataset, all values are averaged, but users should be cautious if analyzing differences of small magnitude.

5. Interpretation of HIC – false positives/negatives: Not every antibody with a high HIC will necessarily fail developability, nor will every antibody with low HIC always be free of issues. HIC specifically catches hydrophobic-driven liabilities. It might miss antibodies that have issues due to other reasons (say, a tendency to unfold at 37°C because of a specific instability that isn’t due to hydrophobic surface). Conversely, an antibody with a hydrophobic paratope might show high HIC, but if that hydrophobic patch is also the antigen-binding site and the antigen in vivo effectively “shields” it (because the antibody is usually bound to target), it might behave well in patients. These context-specific cases are exceptions, but they exist. Moreover, some engineered antibodies (like those with mutations in constant domains for half-life or stability) might behave oddly in HIC if those mutations alter the protein’s surface properties in unforeseen ways. For example, an IgG4 which tends to dissociate into half-molecules might show multiple peaks in HIC. Indeed, Engin Yapici notes IgG4s had lower monomer percentage and stability in the dataset
medium.com
, so any systematic differences in subclass might also reflect in HIC slightly (if IgG4 tends to have more flexible CH3 domain, perhaps minor effects). In general, HIC is considered a robust primary screen, but it should be interpreted alongside other assays. GDPa1’s multi-assay approach acknowledges this: they identify clusters of related assays
pubmed.ncbi.nlm.nih.gov
, so HIC should be looked at in context (e.g. if an antibody has high HIC but normal AC-SINS and polyreactivity, it might not be as risky as one that is high in all three).

6. Confounding by multi-specific formats or conditional stability: The GDPa1 set is mainly monospecific IgGs, so this is less an issue there. But it’s worth noting that the “hydrophobicity” measured is for the whole antibody. If an antibody is formulated or behaves as a dimer (self-associating), the effective hydrophobic surface area changes (some surfaces become buried in self-interaction). So a strongly self-interacting antibody might paradoxically elute a bit earlier if dimers form that have fewer total hydrophobic sites exposed (this is speculative, but in principle possible). Also, some antibodies might interact with the column in specific orientations—e.g. Fab-first vs Fc-first binding to the hydrophobic resin. If the Fc has some hydrophobic patch (perhaps near the CH2 if not fully glycosylated) it could influence retention. Most evidence suggests variable region properties dominate HIC for IgGs
academic.oup.com
, but one can’t entirely ignore constant regions. A difference in constant region (IgG1 vs IgG4, or presence of mutations) could shift retention slightly.

In conclusion, the HIC assay as implemented in GDPa1 is a high-quality, reproducible method for assessing antibody hydrophobicity at scale
biorxiv.org
. However, users of the data and models should be aware of limitations: retention time is not a pure measure of hydrophobicity in all cases (extreme charge, glycosylation, or partial unfolding can skew it
tandfonline.com
), and the assay’s own dynamic range and resolution impose some uncertainty for close values. The GDPa1 preprint authors themselves highlight assay reproducibility challenges and the need for standardization across datasets
pubmed.ncbi.nlm.nih.gov
. They recommend including control antibodies and transparent methods so that others can calibrate their in silico predictions accordingly
pubmed.ncbi.nlm.nih.gov
. The high-throughput nature means minor variants of antibodies (say differing only in a few residues) can be systematically analyzed, but also means each measurement may have less human oversight (thus relying on statistical averaging to weed out anomalies). Despite these caveats, HIC in GDPa1 has been shown to cluster meaningfully with other measures (like SMAC, polyreactivity)
biorxiv.org
 and provides a strong signal for machine learning models. It remains one of the cornerstone assays for developability, with the understanding that it captures the hydrophobicity dimension of the multi-dimensional developability space.

\section{Mitigation Strategies for HIC-Identified Liabilities}
When an antibody shows a high HIC retention (indicating problematic hydrophobicity), the next step is often to consider protein engineering strategies to mitigate this liability. Several realistic approaches can reduce surface hydrophobicity or otherwise improve developability without compromising antigen binding:

1. Targeted Mutations in CDRs: Since hydrophobic patches often originate in the CDR loops (especially H3), a common strategy is to mutate one or more residues in the problematic region to more hydrophilic amino acids. For example, if a CDR-H3 has a stretch of bulky hydrophobics like WWV or LFF, one might mutate a residue in the center to a polar or charged amino acid (e.g. Ser, Thr, Asp) to break the patch. This must be done carefully to preserve binding affinity, but often the periphery of a paratope can be altered without ablating function. In cases where affinity maturation introduced hydrophobic mutations, sometimes reverting a subset of those to germline residues can reduce hydrophobicity while retaining adequate affinity
sciencedirect.com
. Interestingly, natural affinity maturation in vivo tends to select against polyreactivity and hydrophobicity
sciencedirect.com
, so engineered antibodies that are highly matured in vitro can sometimes be “de-humanized” slightly to remove excessive hydrophobic mutations. Many studies have reported single-point substitutions that dramatically lowered aggregation and self-association of antibodies. For instance, replacing a tryptophan that protrudes on the surface with a tyrosine or even a charged residue can reduce a molecule’s HIC retention and aggregation without major impact on stability (tyrosine is still somewhat hydrophobic but less so and more soluble in many contexts). As a concrete example, in one problematic antibody a variable-domain glycosylation was removed, which increased hydrophobic polyreactivity; the solution was to identify nearby aromatic residues and mutate a couple of them to glutamine and lysine, which compensated for the lost glycan shielding
europepmc.org
. This kind of local “patch disruption” by mutation is a direct way to mitigate HIC liabilities.

2. Adjusting Surface Charge: Increasing the overall hydrophilicity isn’t only about removing hydrophobes; it can also be achieved by adding charged or polar residues on the surface. Introducing a charged residue into a hydrophobic patch can have a two-fold benefit: the charge itself improves solubility and repulsion, and it may also induce local structural shifts that expose less hydrophobic area. Some successful engineering efforts to reduce viscosity used this principle, adding an acidic residue on a heavy-chain framework region that participates in self-interaction networks. The Developability Index concept (hydrophobicity + net charge)
frontiersin.org
 implies that one can offset a high SAP (hydrophobic patch score) by boosting the molecule’s net charge. Practically, this could mean mutating a neutral surface residue to a lysine or glutamate. However, one must avoid creating new liabilities (for example, too many positive charges could increase heparin binding and clearance risk). Still, modest increases in negative charge in particular (since most antibodies have basic pI) can improve colloidal stability. A study from Genentech observed that antibodies with moderately basic pI tended to have the best overall profiles, avoiding extremes
scispace.com
. So, if an antibody is extremely basic (pI >9) and also hydrophobic, adding a few acidic residues can both lower pI and disrupt hydrophobic interactions. An empirical approach is alanine or lysine scanning of solvent-exposed hydrophobic residues: replacing each with Ala (to remove bulky side-chain contacts) or Lys (to add charge) and testing developability. Lysine scanning in particular has been used as an “arginine patch” mitigation in some antibodies, replacing an Arg–Tyr–Phe patch with Lys–Tyr–Phe for instance, which increased solubility. Overall, judicious addition of charges on or near hydrophobic hotspots is a proven strategy to reduce self-association.

3. Glycosylation Engineering: Though unconventional, introducing an N-linked glycan in the variable region can dramatically increase hydrophilicity and reduce polyreactivity. In some naturally occurring antibodies, somatic hypermutation introduces an N-X-S/T motif in a CDR loop, leading to a glycan; these glycans often decrease binding to unrelated antigens (i.e. reduce polyreactivity) without ruining specific binding, essentially by sterically and chemically shielding the paratope
pmc.ncbi.nlm.nih.gov
. Protein engineers have mimicked this by deliberately inserting N-linked glycosylation sites into problematic antibodies. If the glycan can be attached in a location that covers a hydrophobic patch (but not in the direct binding interface for the target), it can substantially lower HIC retention and improve solubility. The downside is added heterogeneity and potential immunogenicity of a new glycan, so this is usually a last-resort strategy. However, it has been successfully used in a few cases reported in the literature, and in one case removing such a glycan (to simplify a molecule) caused a surge in polyreactivity that had to be fixed by other mutations
europepmc.org
. This underscores how effective the glycan was at masking hydrophobic surfaces. Thus, glyco-engineering can be viewed as an extension of increasing polarity—attaching a large hydrophilic moiety directly onto the antibody.

4. Framework and Isotype Engineering: If the problematic hydrophobicity stems not just from CDRs but from the antibody’s framework or constant domains, one can consider swapping those. For example, certain heavy-chain germlines have residues in FW3 that create a patch. Choosing a different germline backbone for humanization could remove that. Likewise, if an IgG1 has an issue in CH3 domain that makes it sticky (rare, but some IgG1 allotypes have hydrophobic residues exposed), switching to IgG4 or vice versa might help. In GDPa1, they noted IgG4s had slightly different profiles
medium.com
, so isotype matters. Even $\kappa$ vs $\lambda$ light chain choice can influence developability. These changes are more drastic (they alter many residues at once), so typically one would try point mutations first. But in early discovery, if a lead antibody is very hydrophobic, re-cloning the same CDRs into a different framework scaffold (a so-called framework swap or humanization to a more developable template) can be effective. This essentially “buys” a better biophysical baseline without changing the paratope. For instance, there are known “developability optimized” germline frameworks (like IGHV12-2 or IGHV3-66 with certain mutations) that some companies use to re-graft CDRs.

5. Reducing Aggregation Nuclei via Stability Engineering: Sometimes an antibody is hydrophobic because it is borderline unstable and transiently unfolds, exposing hydrophobic cores. In such cases, improving its stability can indirectly reduce effective hydrophobic exposure. Strategies include introducing disulfide bonds (if a CDR loop is flexible, tethering it might reduce exposure of a patch), or stabilizing secondary structure (proline substitutions in flexible regions to reduce unfolding). If an antibody has a known aggregation-prone region (like a particular strand in a domain that tends to self-associate when partially unfolded), point mutations that increase its local stability can prevent that from happening. These strategies overlap with improving thermal stability (Tm) which might be measured by nanoDSF in GDPa1
medium.com
medium.com
. While raising Tm doesn’t always reduce HIC, there are cases where making an antibody more rigid or stable leads to less aggregation in HIC’s high-salt environment, hence a narrower peak and perhaps slightly shorter retention (since there’s less “stickiness” from transiently exposed hydrophobes). For example, if a mutation removes a cavity or strengthens a hydrophobic core (increasing conformational stability), the antibody might be less prone to the kind of partial unfolding that would otherwise expose hydrophobic sites during the HIC process
academic.oup.com
.

6. Polyreactivity-Specific Mitigations: If HIC is high due to a hydrophobic paratope that also causes polyreactivity, one can attempt to increase specificity by altering that paratope. Affinity maturation often weeds out polyreactive clones in vivo
sciencedirect.com
. In vitro, one can do counter-screens (e.g. ensure the antibody doesn’t bind to irrelevant antigens or membranes) and then introduce mutations that reduce those off-target interactions. These often coincide with reducing hydrophobicity or increasing polarity in the paratope. In one study, simply sulfating a tyrosine in a CDR reduced polyreactivity by adding negative charge
pmc.ncbi.nlm.nih.gov
, although chemical modifications like that are not typical therapeutic solutions (due to heterogeneity). Instead, the equivalent could be glutamate mutation. The key is to break the non-specific binding without losing specific binding. Computational tools now exist that suggest potential mutations to reduce aggregation or polyreactivity — for example, CamSol and Solubis scores highlight residues that contribute to low solubility; mutating those to predefined more soluble residues is a strategy. Some of those suggestions have been experimentally validated to lower HIC retention and improve manufacturability.

It is worth noting that any engineering strategy must consider the \textbf{trade-offs}: binding affinity, potency, and immunogenicity should ideally remain unaffected. The best case is a mutation in a framework region or an outer CDR face that doesn’t contact antigen but lies in the hydrophobic patch area; this can often be changed with minimal impact on affinity. If the hydrophobic patch is right at the binding interface, it’s trickier — one may have to tolerate a slight affinity loss or try more conservative changes (e.g. replace a Leu with a Thr rather than with a charged Asp, to keep some hydrophobic character for binding but add a hydrogen-bonding capability to water). Empirically, a few mutations are usually sufficient. For instance, researchers from MedImmune described “hydrophobic patch silencing” where introducing 2–3 polar mutations on a heavy chain reduced nonspecific binding dramatically without needing to re-affinity-mature the antibody
frontiersin.org
.

In summary, high HIC retention is a warning sign, but there are multiple avenues to “rescue” an antibody: reducing CDR hydrophobicity directly, adding charges or glycans to mask hydrophobic regions, swapping frameworks, or enhancing stability so less hydrophobic surface is exposed. These strategies have been demonstrated in various case studies
academic.oup.com
frontiersin.org
. For the GDPa1 dataset antibodies, one could imagine if an antibody scored poorly (high HIC, high AC-SINS), an engineer might consult the sequence and structure, identify a hydrophobic patch on a CDR, and try a set of mutations to diminish it. With modern high-throughput DNA synthesis, one can even make a small library of variants and re-test them in the same HIC assay to see improvement. In fact, the availability of a predictive model (like a learned HIC predictor) allows for computationally proposing mutations that lower the predicted HIC value
academic.oup.com
, then those can be experimentally confirmed. This iterative design cycle is greatly facilitated by data like GDPa1. The end goal is to ensure that promising antibodies are not discarded solely due to developability issues, but can be engineered to meet developability criteria without compromising their therapeutic function.

\section{Integration of HIC with Other Developability Readouts in GDPa1}
Developability assessment is inherently multi-dimensional. A strength of the GDPa1 dataset is that it provides multiple readouts for each antibody
huggingface.co
. Here we discuss how HIC (hydrophobicity) relates to and complements other key assays in the panel, such as AC-SINS (affinity-capture self-interaction), polyreactivity (CHO cell membrane and ovalbumin binding), solubility/colloidal stability (SMAC), thermal stability (DSF/Tm), and others. Understanding these relationships helps in building holistic models and in making go/no-go decisions for antibody leads.

As noted earlier, HIC correlates strongly with \textbf{SMAC colloidal stability} results (Spearman $\rho\approx0.78$)
biorxiv.org
. SMAC is effectively an orthogonal chromatographic method where antibodies bind to a non-specific surface monolayer; longer retention means more surface “stickiness,” which inversely correlates with solubility
researchgate.net
. The high correlation between HIC and SMAC indicates that they are largely probing the same property—hydrophobic surface stickiness. Thus, these two assays form one cluster of developability attributes. In Jain et al. (2023)’s correlation map of in vitro assays, hydrophobic interaction (HIC) and nonspecific binding assays clustered together
pubmed.ncbi.nlm.nih.gov
. For practical purposes, one might not need both in a screening funnel if one is confident, but the combination can increase confidence (since a false positive in one is unlikely to also be false in the other given their correlation). In any case, an antibody flagged by HIC in GDPa1 is usually also flagged by SMAC (low colloidal stability), and vice versa
biorxiv.org
.

The relationship between HIC and \textbf{self-association measured by AC-SINS} is also important. AC-SINS (Affinity Capture Self-Interaction Nanoparticle Spectroscopy) measures how prone an antibody is to self-associate by detecting a shift in plasmon wavelength when antibodies on a gold nanoparticle bind to free antibodies in solution. In GDPa1, AC-SINS was done at pH 7.4 (and they also mention a pH 6.0 variant)
openreview.net
. HIC and AC-SINS both reflect “stickiness,” but AC-SINS specifically requires two antibodies to interact in solution, whereas HIC is one antibody interacting with a surface. We expect a moderate correlation: indeed, Arsiwala et al. chose HIC and AC-SINS as two separate targets for generative design because each had limited predictive performance, implying they are related but not redundant
openreview.net
. In practice, many antibodies with high HIC also show a significant AC-SINS signal (large $\Delta \lambda$ max shift). Engin’s summary notes that AC-SINS strongly correlates with another self-interaction metric, DLS $k_d$ (the interaction parameter from dynamic light scattering)
medium.com
. This means AC-SINS is a good proxy for high-concentration behavior (viscosity). HIC correlates with those too, but perhaps not as strongly as AC-SINS does, because HIC is measured at dilute conditions and immobilized context. If we consider the data, one might find a correlation (Spearman) between HIC and AC-SINS in GDPa1 on the order of 0.4–0.6 (to be determined from the dataset). Both are positively correlated: more hydrophobic (high HIC) tends to mean more self-interacting (high AC-SINS shift). However, there are outliers: e.g. an antibody might have a peculiar self-association mechanism (maybe via a specific Fab-Fab interaction site) that gives high AC-SINS but not a particularly hydrophobic surface overall. Conversely, an antibody might bind strongly to hydrophobic surfaces (high HIC) but if those patches are located in such a way that two antibodies can’t easily approach each other (steric hindrance), AC-SINS might be lower than expected. These differences justify measuring both. In decisions, if an antibody triggers both high HIC and high AC-SINS flags, it’s almost certainly problematic (likely to aggregate and be viscous)
medium.com
medium.com
. If it’s high in one but not the other, the issue might be context-dependent; for example, maybe it sticks to surfaces (HIC) but not to itself much — that could mean polyreactivity rather than self-aggregation, or vice versa.

HIC is also linked to \textbf{polyreactivity assays} directly. In GDPa1, polyreactivity was assessed by binding to a complex CHO cell membrane protein mixture (CHO-SMP) and to a model protein (ovalbumin)
huggingface.co
. These are reported as Polyreactivity scores (PR) for CHO and OVA. Engin notes that the CHO and ovalbumin polyreactivity assays correlate well with each other
medium.com
, which implies they measure a general tendency of an antibody to bind nonspecifically. HIC should correlate with these as well, since a hydrophobic antibody is more likely to stick to membrane proteins or to hydrophobic patches on ovalbumin. Arsiwala’s preprint likely reported a cluster where CHO and OVA polyreactivity grouped, and HIC/SMAC was another group, with some moderate correlation connecting the groups (perhaps hydrophobicity contributes to polyreactivity but is not the sole factor—polycationic charge can also cause polyreactivity to cells). The MAbs 2023 review by Jain et al. explicitly states that in vitro polyspecificity and hydrophobicity flags were the most predictive of developability issues
pubmed.ncbi.nlm.nih.gov
, indicating both are crucial and somewhat complementary. Polyspecificity assays might catch antibodies that are very sticky due to positively charged patches (they will bind negatively charged membrane components) even if they are not extremely hydrophobic. HIC, on the other hand, might catch an antibody that is hydrophobic but maybe net neutral (which could slip through a charge-based polyspecificity screen). Thus, using both assays covers more ground. A case in point: some antibodies can have high heparin binding (strong polyreactivity due to positive charge) but average HIC (because they aren’t overall hydrophobic). Others have high HIC but low heparin binding. Each would be flagged by one assay and not the other. In GDPa1, they did include a heparin (HAC) assay
huggingface.co
 which specifically measures binding to a highly negatively charged column. Heparin binding (often considered a surrogate for polyreactivity to DNA, etc.) tends to correlate inversely with HIC in some datasets
pmc.ncbi.nlm.nih.gov
 — essentially, antibodies often fall into either “high hydrophobicity” or “high positive charge” liability, rarely both at once, because extremely hydrophobic antibodies often have lower overall charge. But a few unlucky ones could have both or multiple issues.

By examining modeling outputs together, one can gain a deeper understanding. For example, a predictive model might output a high HIC risk score and also a high AC-SINS risk score for a given sequence. In the guided generation experiment, Arsiwala et al. used both HIC and AC-SINS predictors to bias new sequences
openreview.net
, acknowledging that an optimal therapeutic should have low values for both
openreview.net
openreview.net
. They noted that guiding by these predictors sometimes led to sequences outside the training distribution, which needed to be filtered for naturalness
openreview.net
. This hints that minimizing one liability too much (e.g. hydrophobicity) could lead to unnatural sequences that maybe have extremely high charge or unusual composition. Therefore, co-optimizing with another metric (AC-SINS) helped maintain balance. In general, multi-objective design in silico requires understanding trade-offs: making an antibody too polar might reduce HIC and AC-SINS but could lower thermal stability or affinity. Indeed, they mention that some developability improvements come at the cost of other features
openreview.net
.

Looking at other readouts: \textbf{thermal stability} (melting temperatures Tm1, Tm2) did not cluster with HIC; they form a separate cluster of conformational stability
medium.com
. It’s quite possible in GDPa1 data that HIC has little to no correlation with Tm. A very hydrophobic antibody isn’t necessarily low melting – it could be very stable (some hydrophobic patches don’t affect the folded stability much) – and a very high Tm antibody could still have a hydrophobic surface. So those are orthogonal properties. This is important: one can’t assume an antibody with good HIC result will also have good stability or vice versa. Therefore, one must check both. Similarly, \textbf{purity/aggregates by SEC} may correlate somewhat with HIC if the aggregates are caused by hydrophobic interactions. For example, if an antibody tends to aggregate even in mild conditions, it likely has high hydrophobicity. Jain et al. (2017) earlier found some correlation between self-interaction parameters, non-specific binding, and % monomer loss under stress
pubmed.ncbi.nlm.nih.gov
pubmed.ncbi.nlm.nih.gov
. In GDPa1, SEC aggregation was measured (though presumably most were monomeric since these are clinical antibodies). If any antibody had poor SEC monomer percentage, chances are it also had a high HIC (since aggregation often stems from hydrophobic interactions). Engin’s piece noted IgG4s had lower % monomer on average
medium.com
, but that’s a subclass effect due to half-molecule exchange, not hydrophobicity per se.

Another synergy is between \textbf{HIC and viscosity predictions}. Viscosity at high concentration is not directly measured in GDPa1 (since those assays require specialized rheology at high mg/mL). But AC-SINS and DLS $k_d$ are proxies for viscosity tendency
medium.com
. The combination of HIC and AC-SINS can be used to predict if an antibody will have high viscosity in formulation
wuxibiologics.com
. WuXi Biologics even advertises that combining those assays helps predict PK profiles
wuxibiologics.com
, since high viscosity and polyreactivity often correspond to fast clearance. Thus, in absence of direct viscosity data, a logical approach is: if an antibody has both high HIC and high AC-SINS, one should expect high viscosity and formulation challenges
medium.com
medium.com
. If one is high and the other moderate, the risk is intermediate. A modeling output might integrate these: e.g. a principal component that is roughly “overall stickiness” might be derived from HIC, AC-SINS, and polyreactivity together, distinguishing antibodies that are well-behaved vs generally sticky.

In terms of linking predictions, if one builds a comprehensive developability model, one might use HIC as an input to predict other outcomes. In fact, logistic models for clinical developability have been built where in vitro assays like HIC and polyreactivity are binary flagged and used to predict clinical progression
pubmed.ncbi.nlm.nih.gov
. These showed that having a HIC flag or polyreactivity flag was more predictive of failure than any in silico metric alone
pubmed.ncbi.nlm.nih.gov
. So, HIC can be seen as not just an endpoint to predict, but also a feature to predict higher-level outcomes (like “will it go into development or be dropped?”). The interplay is complex: an antibody might pass HIC but fail something else (like expression yield), but at least it reduces one major risk.

Another link is with \textbf{titer (expression)}. While not obvious, extremely hydrophobic antibodies can sometimes be prone to low expression or poor secretion, because they might interact with cellular membranes or quality control chaperones. GDPa1 measured titer by a protein A capture (Valita assay)
huggingface.co
. It would be interesting to see if any correlation exists between HIC and titer. Possibly not strong, but if an antibody is so hydrophobic that it aggregates intracellularly or in the secretory pathway, it could reduce yield. This might show up as a weak inverse correlation between HIC and titer. For instance, an antibody with very high HIC (very hydrophobic) might have slightly lower expression in ExpiCHO or HEK, due to more getting degraded or aggregated. However, many hydrophobic antibodies do express fine, so this might be a weaker trend.

Finally, how do integrated models use these links? A multi-output model could be trained to predict all assays from sequence. It might internally learn a latent representation of the antibody that captures general developability. For example, a latent factor might correspond to “hydrophobic/polyreactive” dimension and would contribute to predictions of HIC, AC-SINS, SMAC, CHO binding, etc., all at once. Another latent factor might correspond to “stability” affecting Tm and aggregation. In the future, one might see holistic developability scores derived from such models. The GDPa1 dataset enables this exploration. Already, Arsiwala et al. demonstrated that including more data (more assays) improved the design of sequences via multi-objective optimization
openreview.net
. The interplay observed was that optimizing one property sometimes hurt another if done in isolation, so a balanced improvement was sought. Realistically, an antibody must clear all developability bars, not just HIC. So if a candidate has good HIC but fails in AC-SINS or vice versa, it still might be problematic. One might apply hierarchical screening: eliminate anything that fails badly in any critical assay. Some companies set threshold “rules” — e.g. HIC $>$ X min or $\Delta\lambda$ $>$ Y nm or polyreactivity score above Z all cause a liability flag
medium.com
. GDPa1 updated such thresholds using 106 approved mAbs as a reference distribution
medium.com
medium.com
. They argue that those thresholds, when combined, give a profile of an antibody. For instance, they might define a “cluster violation” if an antibody exceeds the 90th percentile in any assay cluster
pubmed.ncbi.nlm.nih.gov
. Tushar Jain’s analysis shows that antibodies that violate multiple clusters have a significantly lower probability of clinical success
pubmed.ncbi.nlm.nih.gov
. In practice, if an antibody had high HIC and high polyreactivity (two violations in the “polyreactivity/hydrophobicity cluster”), it would be considered high risk
pubmed.ncbi.nlm.nih.gov
pubmed.ncbi.nlm.nih.gov
. If it only violates one, perhaps medium risk. If none, low risk.

In conclusion, HIC should not be viewed in isolation. Its value is amplified by considering it alongside other developability metrics. In GDPa1, HIC provides a hydrophobicity axis that overlaps strongly with non-specific binding and self-association axes
biorxiv.org
medium.com
. Together, these define the “stickiness” profile of the antibody. Meanwhile, separate axes like conformational stability (Tm) and purity/aggregation or immunogenic sequence liabilities (not measured here) also need attention. The best outcome is an antibody with low HIC retention, low AC-SINS shift, low polyreactivity score, high thermal stability, high monomer purity, etc. The worst would be the opposite on all. Most antibodies lie in between, with some strengths and some weaknesses. By linking modeling outputs for HIC with those for other assays, one can pinpoint specific issues. For example: a model might predict an antibody has moderately high HIC but low heparin binding; this suggests hydrophobic-driven polyreactivity rather than charge-driven. Another might predict high heparin, low HIC, meaning a cationic problem. Each scenario suggests different engineering fixes (the first needs hydrophobic patch mutation, the second might need reducing positive charges). Thus, the multi-assay view allows tailored mitigation strategies. This integrated perspective is exactly what datasets like GDPa1 are meant to facilitate, and the information gleaned can ultimately guide both antibody discovery (by prioritizing inherently better-behaved sequences) and antibody engineering (by highlighting which property needs improvement and by how much).

\section{Conclusion}
Hydrophobic Interaction Chromatography is a sensitive and high-throughput probe of an antibody’s surface hydrophobicity, and in the context of the GDPa1 developability dataset it has proven to be a linchpin assay connecting to many facets of antibody liabilities. A long HIC retention time in GDPa1 clearly reflects the presence of exposed hydrophobic patches on the antibody, which in turn correlates with a heightened risk of aggregation, self-association (high viscosity), polyspecific binding, and in vivo clearance issues
medium.com
pmc.ncbi.nlm.nih.gov
. We have detailed how HIC works and why it matters: it essentially condenses complex surface properties into a single measurable metric that flags “stickiness.” We examined various approaches to predict and model HIC outcomes, from simple sequence rules (e.g. CDR hydrophobicity scores and charge patches)
tandfonline.com
pmc.ncbi.nlm.nih.gov
 to advanced structure-based descriptors (like calculated hydrophobic patch areas and SAP scores)
pubmed.ncbi.nlm.nih.gov
 and machine learning models that integrate multiple features
academic.oup.com
. These modeling efforts, empowered by data such as GDPa1, are increasingly accurate and can be used not only to forecast an antibody’s behavior but to assist in designing improved variants.

We have also emphasized that HIC, like any assay, has its limits: the assay conditions (high salt, specific resin) mean that extremely high or low retentions must be interpreted with context, and factors like charge can confound results
tandfonline.com
. Nonetheless, within the GDPa1 platform, HIC was reproducible and aligned strongly with orthogonal measures of colloidal stability
biorxiv.org
. By understanding these nuances, researchers can better utilize HIC data—knowing when a high retention is a true red flag versus when it might be mitigated by other factors.

In terms of mitigation, we outlined concrete protein engineering strategies to reduce HIC-defined liabilities: targeted CDR mutations to disrupt hydrophobic patches, introduction of polar or charged residues to increase solubility, occasional use of glycosylation to shield problematic surfaces, and other framework/stability adjustments
academic.oup.com
frontiersin.org
. These strategies have been validated in the literature and offer routes to rescue antibodies that otherwise might be untenable due to developability issues. Importantly, the choice of strategy should be informed by a holistic view of the antibody’s profile—something GDPa1 enables by providing multiple assay readouts for the same molecules.

Finally, we discussed how HIC predictions and results dovetail with other developability metrics like AC-SINS self-interaction, polyreactivity assays, thermal stability, etc. HIC does not stand alone; it is part of a constellation of assays that together define an antibody’s “developability signature.” In GDPa1, HIC showed especially strong connection to assays measuring non-specific interactions (SMAC, polyreactivity)
biorxiv.org
, reinforcing that hydrophobicity-driven liabilities often coincide. By integrating these data, one can achieve a more reliable assessment than any single assay could provide. For example, the combination of a HIC flag and an AC-SINS flag is highly predictive of high viscosity and poor PK
medium.com
, guiding risk assessment more powerfully than either alone.

The GDPa1 dataset and the PROPHET-Ab platform represent a significant step forward in quantitatively characterizing these relationships across many antibodies
ginkgo.bio
. The dense citation of recent findings in this report illustrates how our understanding has converged: high hydrophobicity is bad news for an antibody drug, but it can be measured, predicted, and often engineered away. As the field progresses, we expect that more refined models (possibly combining sequence, structure, and even experimental feedback) will allow early elimination or correction of problematic antibodies. This will lead to downstream cost savings and higher success rates in antibody development. HIC will undoubtedly remain a cornerstone assay in this endeavor, reflecting as it does a fundamental biophysical property of proteins. Through careful interpretation and in conjunction with complementary assays, HIC data will continue to provide critical insights into which antibody candidates are built for the long haul of clinical development and which may need a bit of molecular makeover before they are ready for prime time.

\medskip
\noindent \textbf{References}

{\small
\begin{enumerate}\itemsep -2pt
\item Ammar Arsiwala \textit{et al}., \textit{“A high-throughput platform for biophysical antibody developability assessment to enable AI/ML model training,”} bioRxiv 2025. 
ginkgo.bio
biorxiv.org

\item Tushar Jain \textit{et al}., \textit{“Identifying developability risks for clinical progression of antibodies using high-throughput in vitro and in silico approaches,”} mAbs 15(1):2200540 (2023). 
pubmed.ncbi.nlm.nih.gov
pubmed.ncbi.nlm.nih.gov

\item Tushar Jain \textit{et al}., \textit{“Prediction of delayed retention of antibodies in hydrophobic interaction chromatography from sequence using machine learning,”} Bioinformatics 33(23):3758-3766 (2017). 
academic.oup.com
academic.oup.com

\item Eliott Park and Saeed Izadi, \textit{“Molecular surface descriptors to predict antibody developability: sensitivity to parameters, structure models, and conformational sampling,”} mAbs 16(1):2362788 (2024). 
pubmed.ncbi.nlm.nih.gov
pubmed.ncbi.nlm.nih.gov

\item Engin Yapici, \textit{“What It Looks Like to Industrialize Antibody Developability Assays,”} Medium (Sep 2025). 
medium.com
medium.com

\item Jonas Teroerde \textit{et al}., \textit{“Optimizing colloidal stability and viscosity of multispecific antibodies at the discovery-development interface: a systematic predictive case study,”} mAbs 17(1): 2153414 (2025). 
tandfonline.com

\item J. S. Obermeyer \textit{et al}., \textit{“Structure-based design of antibodies with high stability and reduced viscosity,”} J. Mol. Biol. 429(7): 1164-1179 (2017). 
frontiersin.org
frontiersin.org

\item M. M. Joubert \textit{et al}., \textit{“The effects of charge mutations on antibody solubility and viscosity,”} mAbs 8(4): 772-782 (2016). 
pmc.ncbi.nlm.nih.gov
pmc.ncbi.nlm.nih.gov

\item L. M. Kamath \textit{et al}., \textit{“Polyreactivity and Polyspecificity in Therapeutic Antibody Development,”} mAbs 14(1):2000406 (2022). 
pmc.ncbi.nlm.nih.gov
pmc.ncbi.nlm.nih.gov

\item K. J. Lauer \textit{et al}., \textit{“Developability Index: A rapid in silico tool for the screening of antibody aggregation propensity,”} J. Pharm. Sci. 101(1):102-115 (2012). 
frontiersin.org

\end{enumerate}
}

\end{document}